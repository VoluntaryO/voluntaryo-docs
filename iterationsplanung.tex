%!TEX program = xelatex
\documentclass{template/document}
\input{template/document.sty}

\newcommand\documentTitle{Iterationsplanung Projekt VoluntaryO}
\newcommand\documentAuthorA{Dominik Freier}
\newcommand\documentAuthorB{Nhat-Nam Le}
\newcommand\documentAuthorC{Philipp Meier}
\newcommand\documentAuthorD{Robin Bader}
\newcommand\documentSubject{Software Engineering 2 - Projekt}

\makeindex

\begin{document}
 
    \input{template/title}

    \tableofcontents
    \newpage

    \section*{Änderungshistorie}
    \begin{table}[H]
        \tablestyle
        \tablealtcolored
        \begin{tabularx}{\textwidth}{l l X r}
        \tableheadcolor
            \tablehead Version & 
            \tablehead Datum & 
            \tablehead Änderung & 
            \tablehead Person \\  
        \tablebody
            v1.0 & 07.03.2014 & Initialisierung & p1meier \tabularnewline
            v1.1 & 08.03.2014 & Iterationsplanung E1 & p1meier \tabularnewline
            v1.2 & 12.03.2014 & Iterationsplanung E2 & p1meier \tabularnewline
            v1.3 & 16.03.2014 & Realisierung E1 & p1meier \tabularnewline
            v1.4 & 25.03.2014 & Iterationsplanung C1, C2, C3, T1 & p1meier \tabularnewline
            v1.5 & 17.04.2014 & Realisierung für E3 eingefügt & p1meier \tabularnewline
            v1.6 & 23.04.2014 & Iterationsplanung für C2 angepasst & p1meier \tabularnewline
            v1.7 & 30.04.2014 & Realisierung für C1 eingefügt & p1meier \tabularnewline
            v1.8 & 12.05.2014 & Realisierung für C2 eingefügt & p1meier \tabularnewline
            v1.9 & 25.05.2014 & Realisierung für T1 eingefügt & p1meier \tabularnewline
        \tableend
        \end{tabularx} 
    \end{table}
    \newpage


    \chapter{Einführung}
	\section{Zweck}
	Mit diesem Dokument werden die Iterationen geplant, sowie überprüft. Pro Iteration ist jeweils zu sehen, wie die Arbeitspakete geplant sind, sowie die tatsächliche Realisierung der Iteration geschah. Die Pakete werden im \href{http://sinv-56086.edu.hsr.ch:40010/}{JIRA} verwaltet. Die beiden Sichten Planung und Realisierung stellen jeweils Snapshots dar. In einer zusätzlichen Sektion werden dann die gewonnenen Erkenntnisse beschrieben.

	\section{Gültigkeitsbereich}
	Dieses Dokument ist während den Phasen Elaboration, Construction, Transition gültig. Die Inception-Phase wurde nicht geplant.
    \chapter{Elaboration E1}
	\section{Planung}
    \begin{table}[H]
        \tablestyle
        \tablealtcolored
        \begin{tabularx}{\textwidth}{l X l r}
        \tableheadcolor
            \tablehead Key &
            \tablehead Summary & 
            \tablehead Component/s &
            \tablehead Aufwand geschätzt h \tabularnewline  
        \tablebody
		    VOLO-35 & Zeitplanung erstellen E1 & Project Management & 3 \tabularnewline
		    VOLO-33 & Projektsitzung E1 & Project Management & 4 \tabularnewline
		    VOLO-24 & Anforderungsspezifikation & Requirements & 3 \tabularnewline
		    VOLO-25 & VOLO-24 Funktionale Anforderungen / Use Case in Brief Format & Requirements & 4 \tabularnewline
		    VOLO-53 & VOLO-24 Nichtfunktionale Anforderungen & Analyse & 1 \tabularnewline
		    VOLO-18 & Logo  & Environment & 4 \tabularnewline
		    VOLO-34 & Machbarkeitsanalyse & Project Management & 4 \tabularnewline
		    VOLO-36 & Meilenstein MS2 Abgabe Projektplan & Project Management & 6 \tabularnewline
		    VOLO-37 & VOLO-36 Resultate aus Review MS2 impementieren & Project Management & 2 \tabularnewline
		    VOLO-51 & Prototype Infrastruktur E1 & Entwicklungsumgebung & 16 \tabularnewline
		    VOLO-8 & Infrastruktur & Projektplan & 0 \tabularnewline
		    VOLO-10 & VOLO-8 TFS Setup & Projektplan & 0 \tabularnewline
		    VOLO-11 & VOLO-8 MSSQL Setup & Projektplan & 0 \tabularnewline
		    VOLO-12 & VOLO-8 JIRA Setup & Environment & 4 \tabularnewline
		    VOLO-13 & VOLO-8 Setup Windows on MacBooks & Projektplan & 8 \tabularnewline
		    VOLO-30 & System Sequenz Diagram for complex operations & Analyse & 4 \tabularnewline
		    VOLO-31 & Operation Contracts for complex use cases & Analyse & 3 \tabularnewline
		    VOLO-50 & Zeitplanung E2 erstellen & Project Management & 2 \tabularnewline
		    VOLO-32 & Datenmodell erstellen & Analyse & 3 \tabularnewline
		    VOLO-28 & GUI Mock-Ups erstellen & Analyse & 3 \tabularnewline
		    VOLO-29 & Personas / Szenarios & Analyse & 3 \tabularnewline
		    VOLO-23 & Learn ASP.net with Tutorial &       & 8 \tabularnewline
		    VOLO-21 & Learn LaTeX &       & 4 \tabularnewline
		    VOLO-7 & Arbeitspakete & Projektplan & 4 \tabularnewline
		    VOLO-19 & Review of Projektplan for Milestone & Projektplan & 5 \tabularnewline
		    \bottomrule
		    \multicolumn{3}{l}{\textbf{Summe}} & 98 \tabularnewline
        \tableend
        \end{tabularx} 
    \end{table}	
	
	\section{Realisierung}
	
	\section{Erkenntnisse}
    %!TEX root = ../../iterationsplanung.tex
\chapter{Elaboration E2}
	\section{Planung}
	
	\section{Realisierung}
	
	\section{Erkenntnisse}
    %!TEX root = ../../iterationsplanung.tex
\chapter{Elaboration E3}
	\section{Planung}
    \begin{table}[H]
        \tablestyle
        \tablealtcolored
        \begin{tabularx}{\textwidth}{l X l l r}
        \tableheadcolor
            \tablehead Key &
            \tablehead Summary & 
            \tablehead Priority &
            \tablehead Component/s &
            \tablehead Estimate [h] \tabularnewline  
        \tablebody
			VOLO-44  & Einrichten Infrastruktur                              & Major & Environment, Implementation & 8 \tabularnewline
			VOLO-42  & VOLO-44 Automatisches Building                        & Minor & Entwicklungsumgebung        & 6 \tabularnewline
			VOLO-38  & Externes Design                                       & Minor & Analyse                     & 6 \tabularnewline
			VOLO-70  & MS4 Ende Elaboration                                  & Major & Projektmanagement           & 4 \tabularnewline
			VOLO-73  & VOLO-70 Resultate aus Review MS4 implementieren       & Major & Projektmanagement           & 3 \tabularnewline
			VOLO-101 & VOLO-70 Risiken neu einschätzen                       & Major & Projektmanagement           & 2 \tabularnewline
			VOLO-57  & Login- und Rollenfunktion                             & Major & Implementation              & 6 \tabularnewline
			VOLO-117 & VOLO-57 Rollensystem                                  & Major & Implementation              & 6 \tabularnewline
			VOLO-58  & Logoutfunktion                                        & Major & Implementation              & 3 \tabularnewline
			VOLO-76  & Benutzer meldet sich für Helfereinsatz an             & Major & Implementation              & 8 \tabularnewline
			VOLO-77  & Benutzer meldet sich für Helfereinsatz ab             & Major & Implementation              & 5 \tabularnewline
			VOLO-80  & Helfereinsätze verwalten CRD                          & Major & Implementation              & 10 \tabularnewline
			VOLO-84  & Events importieren (Verbands API)                     & Major & Implementation              & 7 \tabularnewline
			VOLO-85  & Mitglieder importieren (webling API)                  & Major & Implementation              & 8 \tabularnewline
			VOLO-121 & VOLO-85 MemberService für Operationen mit Mitgliedern & Major & Implementation              & 15 \tabularnewline
			VOLO-81  & Events bearbeiten                                     & Major & Implementation              & 10 \tabularnewline
			VOLO-118 & VOLO-81 Events Mannschaft zuordnen                    & Major & Implementation              & 2 \tabularnewline
			VOLO-61  & Events anzeigen                                       & Major & Implementation              & 6 \tabularnewline
			VOLO-90  & Integrationstest E3                                   & Major & Qualitätsmanagement         & 5 \tabularnewline
			VOLO-63  & Reserve Bugfixing E3                                  & Major & Qualitätsmanagement         & 6 \tabularnewline
			VOLO-66  & Projektsitzung E3                                     & Major & Projektmanagement           & 8 \tabularnewline
			VOLO-105 & Exception Handling/Error Reporting                    & Major & Qualitätsmanagement         & 8 \tabularnewline
			VOLO-108 & Test Coverage erhöhen                                 & Major & Qualitätsmanagement         & 6 \tabularnewline
			VOLO-109 & Code Review E3 für wesentliche Komponenten            & Major & Qualitätsmanagement         & 8 \tabularnewline
		    \bottomrule
		    \multicolumn{4}{l}{\textbf{Summe}} & 156 \tabularnewline
        \tableend
        \end{tabularx} 
    \end{table}	

   	\subsection{Bemerkung Planung E3}
   	Die Phase E2 wurde frühzeitig beendet. Daher gibt es für diese Phase ein höheres Studensaldo.
	
	\section{Realisierung}
	
	\section{Erkenntnisse}
    %!TEX root = ../../iterationsplanung.tex
\chapter{Construction C1}
	\section{Planung}
    \begin{table}[H]
        \tablestyle
        \tablealtcolored
        \begin{tabularx}{\textwidth}{l X l l r}
        \tableheadcolor
            \tablehead Key &
            \tablehead Summary & 
            \tablehead Priority &
            \tablehead Component/s &
            \tablehead Estimate [h] \tabularnewline  
        \tablebody
			VOLO-57  & Loginfunktion                              & Major & Implementation      & 6 \tabularnewline
			VOLO-58  & Logoutfunktion                             & Major & Implementation      & 3 \tabularnewline
			VOLO-103 & Gestaltung Layout                          & Major & Implementation      & 8 \tabularnewline
			VOLO-84  & Events importieren (Verbands API)          & Major & Implementation      & 7 \tabularnewline
			VOLO-85  & Mitglieder importieren (webling API)       & Major & Implementation      & 8 \tabularnewline
			VOLO-90  & Integrationstest C1                        & Major & Qualitätsmanagement & 5 \tabularnewline
			VOLO-63  & Reserve Bugfixing C1                       & Major & Qualitätsmanagement & 6 \tabularnewline
			VOLO-66  & Projektsitzung C1                          & Major & Projektmanagement   & 8 \tabularnewline
			VOLO-81  & Helfereinsätze/Events Mannschaft zuordnen  & Major & Implementation      & 10 \tabularnewline
			VOLO-105 & Exception Handling/Error Reporting         & Major & Qualitätsmanagement & 8 \tabularnewline
			VOLO-108 & Test Coverage erhöhen                      & Major & Qualitätsmanagement & 6 \tabularnewline
			VOLO-109 & Code Review C1 für wesentliche Komponenten & Major & Qualitätsmanagement & 8 \tabularnewline
		    \bottomrule
		    \multicolumn{4}{l}{\textbf{Summe}} & 83 \tabularnewline
        \tableend
        \end{tabularx} 
    \end{table}	
	
	\section{Realisierung}
	
	\section{Erkenntnisse}
    %!TEX root = ../../iterationsplanung.tex
\chapter{Construction C2}
	\section{Planung}
	
	\section{Realisierung}
	
	\section{Erkenntnisse}
    %!TEX root = ../../iterationsplanung.tex
\chapter{Transition T1}
	\section{Planung}
    \begin{table}[H]
        \tablestyle
        \tablealtcolored
        \begin{tabularx}{\textwidth}{l X l l r}
        \tableheadcolor
            \tablehead Key &
            \tablehead Summary & 
            \tablehead Priority &
            \tablehead Component/s &
            \tablehead Estimate [h] \tabularnewline  
        \tablebody
			VOLO-112 & Persönliche Schlussberichte erstellen & Major & Environment       & 8 \tabularnewline
			VOLO-86  & Abgabedokumente vorbereiten           & Major & Environment       & 20 \tabularnewline
			VOLO-87  & Erstellung der Projektpräsentation    & Major & Environment       & 6 \tabularnewline
			VOLO-88  & MS6 Schlusspräsentation/-abgabe      & Major & Environment       & 4 \tabularnewline
			VOLO-69  & Projektsitzung T1                     & Major & Projektmanagement & 4 \tabularnewline
		    \bottomrule
		    \multicolumn{4}{l}{\textbf{Summe}} & 42 \tabularnewline
        \tableend
        \end{tabularx} 
    \end{table}	
	
	\section{Realisierung}
	
	\section{Erkenntnisse}
	

    % \input{template/playground}
    
    
\end{document}
