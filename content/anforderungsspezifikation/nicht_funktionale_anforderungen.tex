\chapter{Nichtfunktionale Anforderungen}
	\section{Qualitätsmerkmale}
	Inhalt dieses Abschnittes ist es, nichtfunktionale Anforderungen an dieses Projekt festzuhalten. Für das Qualitätsmodell verwenden wir die Software-Qualitätsmerkmale der Norm ISO 9126.
	\subsection{Funktionalität(functionality)}
	\begin{table}[H]
    	\tablestyle
    	\tablealtcolored
    	\begin{tabularx}{\textwidth}{l X l}
        	\tablebody
          	\textbf{Richtigkeit (accuracy)} & Zeitlich können die Helfereinsätze im 5-Minuten-Raster gesetzt werden. Mitglieder können sich nicht in Helferinsätze einschreiben, wenn zu diesem Zeitpunkt bereits ein eingeschriebener Helfereinsatz existiert.
            \tabularnewline
        	\textbf{Interoperabilität (interoperability)} & Für die Webplattform soll die aktuellste Version von Google Chrome verwendet werden (Stand: März 2014). 
            \tabularnewline
         	\textbf{Sicherheit (security)} & Neben der Aufteilung von verschiedenen Benutzerrollen (Admin, Planer, Mitglied), erhält jedes Mitglied seinen eigenen passwortgeschützten Benutzeraccount. Rechte der folgenden Benutzerrollen:
     
     
         		\begin{itemize}
					\item \textbf{Admin:} 	Besitzt alle Rechte, verwaltet überwiegend Benutzermutationen
					\item \textbf{Planer:} 	Verwaltet die Events und kann ebenfalls Helfereinsätze konfigurieren.		
					\item \textbf{Mitglied:}	Schreibt sich in die vorgegebenen Helfereinsätze einen Events ein.
				\end{itemize}
            \tabularnewline
        	\tableend
    	\end{tabularx}
   		\caption{Funktionalität}
	\end{table}
	
	
	\subsection{Zuverlässigkeit (reliability)}
	\begin{table}[H]
    	\tablestyle
    	\tablealtcolored
    	\begin{tabularx}{\textwidth}{l X l}
        	\tablebody
          	\textbf{Fehlertoleranz (fault tolerance)} & Gemäss der Fehlertoleranz bestehen keine besonderen Anforderungen. 
            \tabularnewline
        	\textbf{Wiederherstellbarkeit (recoverability)} & Durch das tägliche Backup der Daten, kann der Zustand des Vortages schnell wiederhergestellt werden.
            \tabularnewline
           	\tableend
    	\end{tabularx}
   		\caption{Zuverlässigkeit}
	\end{table}

	
	\subsection{Benutzbarkeit (usability)}
	\begin{table}[H]
    	\tablestyle
    	\tablealtcolored
    	\begin{tabularx}{\textwidth}{l X l}
        	\tablebody
          	\textbf{Erlernbarkeit (learnability)} & Organisatoren (Rolle: Planer) erhalten zu beginn eine kurze Schulung über die Verwaltung von Events und Helfereinsätzen. Sie erhalten zugleich die benötigten Informationen, um als Anlaufstelle für die restlichen Benutzer (Rolle: Mitglied) im Verein dienen zu können.
            \tabularnewline
        	\textbf{Bedienbarkeit (operability)} & Für einen Helfer (Rolle: Mitglied) sollte die Anmeldungsroutine für ein Event innerhalb von 2 bis 3 Minuten abgewickelt werden können.\tabularnewline
        	\tableend
    	\end{tabularx}
   		\caption{Benutzbarkeit}
	\end{table}

	
	\section{Leistung (performance)}
	Das Zeitverhalten hängt stark von der Hardwareumgebung ab. Da in unserem Projekt die endgültige Umgebung noch nicht feststeht, können keine definitiven Angaben zum Zeitverhalten gegeben werden. Sobald die Serverumgebung vollständig geladen ist, soll die Operation "Einschreiben für einen Helfereinsatz" inklusive Validierung innerhalb von 3 Sekunden abgewickelt werden können.

	\subsection{Mengenanforderungen (quantities)}
	Im Rahmen des Projekts sollten rund 250 Benutzer verwaltet werden. Im Verein sind mehrere Mannschaften mit durchschnittlich 15 bis 20 Mitglieder. 
	Zu Spitzenzeiten rechnen wir mit 30 gleichzeitig aktiven Sitzungen (kurz vor Ablauf der Sperrzeit). \\Ein durchschnittlicher Event beinhaltet 10 bis 15 freie Helfereinsätze, diese Zahl kann jedoch nach Eventgrösse variieren. Pro Woche werden durchschnittlich zwei Events durchgeführt.
	\\Die Datenbank sollte eine Grösse von maximal 50-100 MB nicht übersteigen.
	
	
	\section{Internationalisierung}
	Wir verzichten auf eine Internationalisierung, weil unser Projekt im Moment sich nur auf einen Verein bezieht (deutschsprachig).
	
	\section{Programmiersprache}
	Die Teammitglieder arbeiten alle mit Visual Studio 2013. Wir verwenden C\# als Programmiersprache. 

	
