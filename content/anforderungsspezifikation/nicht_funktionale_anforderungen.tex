\chapter{Nicht Funktionale Anforderungen}
	\section{Checkliste}
	\begin{itemize}
		\item Qualitätsmerkmale (Funktionalität, Zuverlässigkeit, Benutzbarkeit,
Effizienz, Änderbarkeit,Übertragbarkeit)
		\item Mengen, Leistung
		
		\item Überprüfbar formuliert
	\end{itemize}
	
	\section{Qualitätsmerkmale}
	Inhalt dieses Abschnittes ist es, nicht-funktionale Anforderungen an dieses Projekt festzuhalten.Für das Qualitätsmodell verwenden wir die Software-Qualitätsmerkmale der Norm ISO 9126.
	\subsection{Funktionalität(functionality)}
	\begin{table}[H]
    	\tablestyle
    	\tablealtcolored
    	\begin{tabularx}{\textwidth}{l X l}
        	\tablebody
        	\textbf{Angemessenheit (suitability)} & <text>
        	\tabularnewline
          	\textbf{Richtigkeit (accuracy)} & <text>
            \tabularnewline
        	\textbf{Interoperabilität (interoperability)} & <text>
            \tabularnewline
        	\textbf{Ordnungsmässigkeit(compliance)} & <text>
            \tabularnewline
         	\textbf{Sicherheit (security)} & <text>
            \tabularnewline
        	\tableend
    	\end{tabularx}
   		\caption{Funktionalität}
	\end{table}
	
	
	\subsection{Zuverlässigkeit (reliability)}
	\begin{table}[H]
    	\tablestyle
    	\tablealtcolored
    	\begin{tabularx}{\textwidth}{l X l}
        	\tablebody
        	\textbf{Reife (maturity)} & <text>
        	\tabularnewline
          	\textbf{Fehlertoleranz (fault tolerance)} & <text>
            \tabularnewline
        	\textbf{Wiederherstellbarkeit (recoverability)} & <text>
            \tabularnewline
        	\tableend
    	\end{tabularx}
   		\caption{Zuverlässigkeit}
	\end{table}

	
	\subsection{Benutzbarkeit (usability)}
	\begin{table}[H]
    	\tablestyle
    	\tablealtcolored
    	\begin{tabularx}{\textwidth}{l X l}
        	\tablebody
        	\textbf{Verständlichkeit (understandability)} & <text>
        	\tabularnewline
          	\textbf{Erlernbarkeit (learnability)} & <text>
            \tabularnewline
        	\textbf{Bedienbarkeit (operability)} & <text>
            \tabularnewline
        	\tableend
    	\end{tabularx}
   		\caption{Benutzbarkeit}
	\end{table}

	
	\subsection{Effizienz (efficiency)}	
	\begin{table}[H]
    	\tablestyle
    	\tablealtcolored
    	\begin{tabularx}{\textwidth}{l X l}
        	\tablebody
        	\textbf{Zeitverhalten (time behaviour)} & <text>
        	\tabularnewline
          	\textbf{Verbrauchsverhalten (resource behaviour)} & <text>
            \tabularnewline
        	\tableend
    	\end{tabularx}
   		\caption{Effizienz}
	\end{table}

	\subsection{Änderbarkeit (maintainability)}
	\begin{table}[H]
    	\tablestyle
    	\tablealtcolored
    	\begin{tabularx}{\textwidth}{l X l}
        	\tablebody
        	\textbf{Analysierbarkeit (analysability)} & <text>
        	\tabularnewline
          	\textbf{Modifizierbarkeit (changeability)} & <text>
            \tabularnewline
          	\textbf{Stabilität (stability)} & <text>
            \tabularnewline
          	\textbf{Prüfbarkeit (testability)} & <text>
            \tabularnewline
        	\tableend
    	\end{tabularx}
   		\caption{Änderbarkeit}
	\end{table}
	
	\subsection{Übertragbarkeit (portability)}
	\begin{table}[H]
    	\tablestyle
    	\tablealtcolored
    	\begin{tabularx}{\textwidth}{l X l}
        	\tablebody
        	\textbf{Anpassbarkeit (adaptability)} & <text>
        	\tabularnewline
          	\textbf{Installierbarkeit (installability)} & <text>
            \tabularnewline
          	\textbf{KonforÄndermität (conformance)} & <text>
            \tabularnewline
          	\textbf{Austauschbarkeit (replaceability)} & <text>
            \tabularnewline
        	\tableend
    	\end{tabularx}
   		\caption{Übertragbarkeit}
	\end{table}
	
	\section{Menge}
	
	\section{Schnittstellen}
	<Beschreibung der Schnittstellen der Software>
	
	\section{Sicherheit}
	