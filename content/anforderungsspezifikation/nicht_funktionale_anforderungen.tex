\chapter{Nichtfunktionale Anforderungen}
	\section{Checkliste}
	\begin{itemize}
		\item Qualitätsmerkmale (Funktionalität, Zuverlässigkeit, Benutzbarkeit,
Effizienz, Änderbarkeit,Übertragbarkeit)
		\item Mengen, Leistung
		
		\item Überprüfbar formuliert
	\end{itemize}
	
	\href{http://de.wikipedia.org/wiki/ISO/IEC_9126}{ISO 9126}
	\section{Qualitätsmerkmale}
	Inhalt dieses Abschnittes ist es, nicht-funktionale Anforderungen an dieses Projekt festzuhalten.Für das Qualitätsmodell verwenden wir die Software-Qualitätsmerkmale der Norm ISO 9126.
	\subsection{Funktionalität(functionality)}
	Die Fähigkeit des Softwareprodukts, ausdrücklich genannte oder implizite Anforderungen zu erfüllen, die für den Gebrauch der Software unter spezifizierten Bedingungen notwendig sind.
	\begin{table}[H]
    	\tablestyle
    	\tablealtcolored
    	\begin{tabularx}{\textwidth}{l X l}
        	\tablebody
        	\textbf{Angemessenheit (suitability)} & \textcolor{red}{<Die Fähigkeit des Softwareprodukts, für die spezifizierten Aufgaben und Ziele des Benutzers einen angemessenen Funktionsumfang zur Verfügung zu stellen. >}
        	\tabularnewline
          	\textbf{Richtigkeit (accuracy)} & Zeitlich können die Helfereinsätze im Viertelstundenraster gesetzt werden. Mitglieder können sich nicht in Helferinsätze einschreiben, wenn zu diesem Zeitpunkt bereits ein eingeschriebener Helfereinsatz existiert.
            \tabularnewline
        	\textbf{Interoperabilität (interoperability)} & Das Projekt VolutaryO stellt eine für den Benutzer plattformunabhängige Weboberfläche dar. Unterschiede in der Darstellung können lediglich bei der Verwendung von unterschiedlichen Webbrowsern auftreten.
            \tabularnewline
        	\textbf{Ordnungsmässigkeit(compliance)} & \textcolor{red}{<Die Eigenschaft der Software, Normen und Vereinbarungen bezüglich des Qualitätsmerkmals „Funktionalität“ zu erfüllen.>}
            \tabularnewline
         	\textbf{Sicherheit (security)} & Neben der Aufteilung von verschiedenen Benutzerrollen (Admin, Planer, Mitglied), erhält jedes Mitglied seinen eigenen passwortgeschützten Benutzeraccount. Rechte der folgenden Benutzerrollen:
     
     
         		\begin{itemize}
					\item \textbf{Admin:} 	Besitzt alle Rechte, verwaltet überwiegend Benutzermutationen
					\item \textbf{Planer:} 	Verwaltet die Events und kann ebenfalls Helfereinsätze konfigurieren.		
					\item \textbf{Mitglied:}	Schreibt sich in die vorgegebenen Helfereinsätze einen Events ein.
				\end{itemize}
            \tabularnewline
        	\tableend
    	\end{tabularx}
   		\caption{Funktionalität}
	\end{table}
	
	
	\subsection{Zuverlässigkeit (reliability)}
	Die Fähigkeit des Softwareprodukts, ein bestimmtes Leistungsniveau unter definierten Bedingungen über einen definierten Zeitraum aufrecht zu erhalten.
	\begin{table}[H]
    	\tablestyle
    	\tablealtcolored
    	\begin{tabularx}{\textwidth}{l X l}
        	\tablebody
        	\textbf{Reife (maturity)} & \textcolor{red}{<Die Fähigkeit des Softwareprodukts, Ausfälle Aufgrund von Fehlerzuständen in der Software zu vermeiden.>}
        	\tabularnewline
          	\textbf{Fehlertoleranz (fault tolerance)} & \textcolor{red}{<Die Fähigkeit des Softwareprodukts, ein bestimmtes Leistungsniveau im Falle von Fehlerzuständen oder Unvereinbarkeit mit der spezifizierten Schnittstelle aufrecht zu erhalten.>}
            \tabularnewline
        	\textbf{Wiederherstellbarkeit (recoverability)} & Durch das tägliche Backup der Web-Applikation, kann der Zustand des Vortages schnell wiederhergestellt werden.
            \tabularnewline
        	\tableend
    	\end{tabularx}
   		\caption{Zuverlässigkeit}
	\end{table}

	
	\subsection{Benutzbarkeit (usability)}
	Benutzbarkeit bezieht sich darauf, dass die Software von bestimmten Nutzergruppen unter bestimmten Bedingungen verstanden, erlernt und benutzt werden kann und subsumiert folgende Untermerkmale:
	\begin{table}[H]
    	\tablestyle
    	\tablealtcolored
    	\begin{tabularx}{\textwidth}{l X l}
        	\tablebody
        	\textbf{Verständlichkeit (understandability)} & <text>
        	\tabularnewline
          	\textbf{Erlernbarkeit (learnability)} & <text>
            \tabularnewline
        	\textbf{Bedienbarkeit (operability)} & Natürlich sollte 
            \tabularnewline
        	\tableend
    	\end{tabularx}
   		\caption{Benutzbarkeit}
	\end{table}

	
	\subsection{Effizienz (efficiency)}	
	Die Fähigkeit des Softwareprodukts, ein bestimmtes Leistungsniveau in Relation zu den Ressourcen unter definierten Bedingungen zu erbringen. 
	\begin{table}[H]
    	\tablestyle
    	\tablealtcolored
    	\begin{tabularx}{\textwidth}{l X l}
        	\tablebody
        	\textbf{Zeitverhalten (time behaviour)} & <text>
        	\tabularnewline
          	\textbf{Verbrauchsverhalten (resource behaviour)} & <text>
            \tabularnewline
        	\tableend
    	\end{tabularx}
   		\caption{Effizienz}
	\end{table}

	\subsection{Änderbarkeit (maintainability)}
	Die Fähigkeit des Softwareprodukts, modifiziert zu werden. Modifikationen können Korrekturen, Verbesserungen oder Anpassungen der Software an Änderungen der Umgebung umfassen, sowie Änderungen der Anforderungen und der funktionalen Spezifikationen.
	\begin{table}[H]
    	\tablestyle
    	\tablealtcolored
    	\begin{tabularx}{\textwidth}{l X l}
        	\tablebody
        	\textbf{Analysierbarkeit (analysability)} & <text>
        	\tabularnewline
          	\textbf{Modifizierbarkeit (changeability)} & <text>
            \tabularnewline
          	\textbf{Stabilität (stability)} & <text>
            \tabularnewline
          	\textbf{Prüfbarkeit (testability)} & <text>
            \tabularnewline
        	\tableend
    	\end{tabularx}
   		\caption{Änderbarkeit}
	\end{table}
	
	\subsection{Übertragbarkeit (portability)}
	Die Fähigkeit des Softwareprodukts, in eine andere Umgebung übertragen zu werden. Die Umgebung kann dabei sowohl die organisatorische Umgebung als auch die Hardware- oder Software-Umgebung sein.
	\begin{table}[H]
    	\tablestyle
    	\tablealtcolored
    	\begin{tabularx}{\textwidth}{l X l}
        	\tablebody
        	\textbf{Anpassbarkeit (adaptability)} & <text>
        	\tabularnewline
          	\textbf{Installierbarkeit (installability)} & <text>
            \tabularnewline
          	\textbf{KonforÄndermität (conformance)} & <text>
            \tabularnewline
          	\textbf{Austauschbarkeit (replaceability)} & <text>
            \tabularnewline
        	\tableend
    	\end{tabularx}
   		\caption{Übertragbarkeit}
	\end{table}
	
	\section{Leistung (performance)}
	<Antwortzeiten, Durchsatzraten>
	<Achtung: sind von Hardwareumgebung und Systembenutzung abhängig
	\section{Mengenanforderungen (quantities)}
	<Anzahl Kundendatensätze>
	<Anzahl gleichzeitige Benutzer>
		
	\section{Randbedingungen (environment)}
	<Einschränkungen bezüglich Realisierung: Programmiersprache, verw. Oracle DB-server>
	<Achtung: keine unnötigen Einschränkungen>

	
