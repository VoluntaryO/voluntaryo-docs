\chapter{Allgemeine Beschreibung}
    
    \section{Funktionen}
    \subsection{Funktionsbeschreibung}
    Es soll eine Webapplikation entstehen, welche die Verwaltung und Planung der Helfereinsätze eines Unihockey Vereins ermöglicht.
    \\Für das Projekt lassen sich die Funktionen in primäre und sekundäre Funktionen einteilen.    
    
    \subsection{Primäre Funktionen}
    \begin{itemize}
        \item \textbf{F01} Planung der Helfereinsätze
        \item \textbf{F02} Unterteilung des Einsatz nach Zeit und Funktion
        \item \textbf{F03} Importieren der Spieldaten von Webservice
        \item \textbf{F04} Manuelles Erfassen von Spieldaten
        \item \textbf{F05} Importieren der Endbenutzer aus \href{http://www.webling.ch}{webling.ch}
        \item \textbf{F06} Manuelles Erfassen von Mitglied
        \item \textbf{F07} Einteilung der Einsätze nach Mannschaft
        \item \textbf{F08} Einschreiben eines Mitglieds für Einsatz
        \item \textbf{F09} Übersicht der Einsätze für Mitglied
        \item \textbf{F10} Erinnerung der Helfer vor Einsatz via E-Mail
    \end{itemize}
    
    \subsection{Optionale Funktionen/Erweiterungen}
    \begin{itemize}
        \item \textbf{F31} Exportfunktion der Termine (Kalender, iCal)
        \item \textbf{F32} Variante für Festlegung Helfereinsätze Stunden / Mitglied
        \item \textbf{F33} Variante für Festlegung Helfereinsätze Studen  / Mannschaft
    \end{itemize}
    
    \subsection{Benutzer Charakteristik}
    \begin{table}[H]
        \tablestyle
        \tablealtcolored
        \begin{tabularx}{\textwidth}{l X l}
            \tablebody
            \textbf{Mitglied} &
                Der eigentliche Endbenutzer wird ein Mitglied des Vereins sein, welcher nicht besonders vertraut mit Webapplikationen ist. 
                \tabularnewline
            \textbf{Planer} &
                Beim Planer kann von einem technisch versierteren Benutzer ausgegangen werden, welcher allenfalls mit einer kurzen Schulung in das System eingeführt werden kann.
                \tabularnewline
            \textbf{Administrator} &
                Es handelt sich um eine Person, welche technisch versiert ist und die Spezifikation des Systems kennt.
                \tabularnewline
            \tableend
        \end{tabularx}
        \caption{Unit Testing}
    \end{table}
