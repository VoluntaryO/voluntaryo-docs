%!TEX root = ../../architekturdokumentation.tex
\chapter{Logische Architektur}
\section{3-Tier-Architektur}
	Um eine möglichst hohe Abstraktion unseres Codes zu erhalten, haben wir uns für eine 3-Tier-Architektur entschieden, die unsere Applikation in die drei Schichten Web, Service und DataAccesLayer (Dal) unterteilt. Alle Datenbankspezifischen Operationen sollen im DAL vorgenommen werden. Die Zugriffe auf externe Schnittstellen über den Service-Layer. Alle Zugriffe aus dem Web sollen über die Web-Layer behandelt werden. 
    \begin{figure}[h]
  		\vspace{-5pt}
    	\centering
		\includegraphics[width=\textwidth]{content/architekturdokumentation/images/LogischeArchitektur.png}
  		\vspace{-20pt}
    	\caption{Logische Architektur}
	\end{figure}

\newpage
\section{VoluntaryO.Dal (Data Access Layer)}
    \begin{figure}[h]
  		\vspace{-5pt}
    	\centering
    	\includegraphics[width=0.7\textwidth]{content/architekturdokumentation/images/VoluntaryO_Dal_Overview.png}
  		\vspace{-20pt}
    	\caption{Klassen in VoluntaryO.Dal}
	\end{figure}
	Jede Model-Klasse implementiert die Schnittstelle \textit{IObjectState}, resp. leitet von der abstrakten Klasse \textit{Entity} ab. Die Verwendung von ASP.NET Identity ist im Kapitel VoluntaryO.Web näher beschrieben.
	
	\subsection{Vergleich für Objekte}
		Einzig für die Member-Klasse wurde die Vergleichsmethode (Equals) überschrieben. Folgende Attribute der Klasse sind relevant:
		\\\begin{itemize}
			\item \textit{UserName}
			\item \textit{Firstname}
			\item \textit{Lastname}
			\item \textit{Birthdate}
		\end{itemize}
		Der Vergleich wurde nur für die Mitgliedsimport-Funktion überschrieben. Für die übrigen Model-Klassen wird die standard Vergleichsmethode beibehalten (Vergleich der Referenz bei Referenztypen).

	\subsection{Repository / Unit of Work Pattern}
		Durch die Verwendung des \href{https://genericunitofworkandrepositories.codeplex.com/}{\textit{Generic Unit of Work \& Repositories Framework}} (in "'Framework"' abgelegt) ergeben sich folgende Vorteile:
		\\\begin{itemize}	
			\item Austauschbarkeit ORM
			\item Testbarkeit
			\item Jeder Request hat eigene Unit of Work (siehe später Dependecy Injection)
			\item Reduktion der Datenbankabfragen, da nur noch über Unit of Work commited wird
			\item Generische Abfragen von Entitäten
		\end{itemize}
		Folgende Konventionen ergeben sich aus für das Projekt:
		\\\begin{itemize}
			\item Jede Entität implementiert \textit{IObjectState}
			\item Repository kann über \textit{partial Classes} erweitern werden
			\item Oder Repository Klassen implementieren \textit{IRepository} und erben von \textit{Repository}
		\end{itemize}

	\subsubsection{Konventionen für eigene Repository Erweiterungen}
		\begin{itemize}
			\item Objekt(-e) abrufen/suchen \textit{Find*}
			\item Objekt(-e) einfügen \textit{Insert*}
			\item Objekt ändern \textit{Update*}
			\item Objekt hinzufügen \textit{Delete*}
			\item Zuweisung zweier Objekte hinzufügen \textit{Assign*}
			\item Zuweisung zweier Objekte entfernen \textit{Unassign*}
		\end{itemize}

	\subsection{Unit Tests mit Effort}
		\href{https://effort.codeplex.com/}{\textit{Effort}} ist ein ADO.Net Provider und erstellt jeweils eine In-Memory Datenbank für das EF. Die temporäre Datenbank ermöglicht uns ein Testen des Codes, ohne Datenbankzugriffe zu mocken.
		\begin{lstlisting}[language=CSharp, caption=Verwendung Effort für Unit Tests in EffortTest.cs, label=lst:effortunittest, firstnumber=1]
			DbConnection Connection = DbConnectionFactory.CreateTransient();
			VoluntaryoContext VoluntaryoContext = new VoluntaryoContext(_Connection);
	    \end{lstlisting}
	    Den einzelnen Unit Tests steht so eine saubere und testbare Datenbank zur Verfügung. Jeder UnitTest erbt von der Abstrakten Klasse \textit{EffortTest}.

	\subsection{Relationales Modell}
	    \begin{figure}[h]
	  		\vspace{-5pt}
	    	\centering
	    	\includegraphics[width=0.7\textwidth]{content/architekturdokumentation/images/edmx.png}
	  		\vspace{-20pt}
	    	\caption{Relationales Modell für VoluntaryO.Dal}
		\end{figure}
		\subsubsection{Mapping}
			Das Mapping wird mit dem \textit{modelBuilder} des EF gesteuert. Bspw.
			\begin{lstlisting}[language=CSharp, caption=Mapping in VoluntaryoContext.cs, label=lst:mappingcontextcs, firstnumber=1]
	// TeamEventMapping
	modelBuilder.Entity<Team>()
	    .HasMany(t => t.Events)
	    .WithMany(e => e.Teams)
	    .Map(mc =>
	    {
	        mc.MapLeftKey("TeamID");
	        mc.MapRightKey("EventID");
	        mc.ToTable("TeamEventMappings");
	    });
		    \end{lstlisting}

\newpage
\section{VoluntaryO.Service}
	Dieses Subprojekt dient als Abstraktionsschicht zwischen den Packages aus dem DAL und deren Verwendung im Web Layer und enthält einige Business Logik. Dadurch wird die Logik in den Actions der Controller reduziert und eine höhere Kohäsion erreicht. Die Aufteilung  von Namespaces und Ordner wurde aufgrund der Zugehörigkeit zu den Entitätsklassen im Domainmodell gewählt. Aufgabenspezifische Service-Komponenten wurden ebenfalls in der obersten Hierarchiestufe des Service-Projekts in einem separaten Ordner untergebracht.
	
	
	\subsection{MemberService}	
		Beim Verwalten und Importieren der Mitglieder muss gelegentlich komplexere Logik ausgeführt werden, die nicht ins Repository im DAL gehört.
		Eine wichtige Aufgabe des MemberService ist das speichern von importierten Mitgliedern in die Datenbank. Dabei können unterschiedliche Szenarien auftreten:
		\\\begin{itemize}	
			\item Es existiert noch kein Mitglied mit dem selben Vornamen, Nachnamen und Geburtsdatum -> Mitglied kann gespeichert werden
			\item Das Mitglied existiert bereits, es haben sich aber keine Angaben geändert (Telefon, Adresse, E-Mail, Mannschaft)
			\item Das Mitglied existiert, jedoch wurden einige Angaben verändert\\
		\end{itemize}
		
		\noindent	
		Im ersten Fall kann das Mitglied einfach gespeichert werden, ebenso im dritten Fall. Sollte aber eine Situation eintreffen, in der ein Mitglied verändert wurde, werden die Änderungen nicht gespeichert. Stattdessen kann über den MemberService auf eine Liste mit diesen Mitgliedern zugegriffen werden. Der Benutzer kann diese Einträge auswählen und explizit überschreiben.
	
	
	\subsection{EventService}
		Der EventService verbindet den EventController mit dem Event- und HelperTaskRepository. Die Logik zum Hinzufügen von Helfereinsätzen und Mannschaften wird hier implementiert.
	
	\subsection{TeamService}
		Bietet Zugriff auf die Repository-Methoden und stellt Erweiterungspunkt für komplexere Logik dar.
	
	\subsection{SkillService}
		Bietet Zugriff auf die Repository-Methoden und stellt Erweiterungspunkt für komplexere Logik dar.
	
	\subsection{HelperTaskService}
		Der HelperTaskService verbindet den HelperTaskController mit dem HelperTaskRepository. Die Logik zum An- und Abmelden für Helfereinsätze und die Überprüfung von Bedingungen wird hier implementiert.
		
	
	\subsection{Imports}
		Für das Importieren von Mitgliedern und Events wird jeweils eine eigene Serviceklasse verwendet. Die abstrakte Klasse \textit{Importer} dient dazu als Basis und schreibt die Methode \textit{executeImport} vor.
		
	    \begin{figure}[h]
	  		\vspace{-5pt}
	    	\centering
	    	 \includegraphics[width=0.9\textwidth]{content/architekturdokumentation/images/ImportPackageDesign.png}
	  		\vspace{-25pt}
			\caption{Importer-Klassen und deren direkte Collaborators}
		\end{figure}
		
		\subsubsection{MemberImporter}
			Der MemberImporter implementiert die Methode \textit{executeImport}, welche das importieren von Mitgliedern startet. Um eine höhere Kohäsion und niedrige Kopplung zu erreichen, wurden die Teilaufgaben ausgelagert:
			\\\begin{itemize}	
				\item MemberLoader - benutzt die webling.ch Schnittstelle und lädt das CSV
				\item MemberParser - liest den CSV-String und speichert die Werte in eine Liste von Hashtables
				\item MemberMapper - gibt Member-Objekte an MemberService zum Speichern weiter\\
			\end{itemize}



		\subsubsection{EventImporter}
			Der EventImporter implementiert ebenfalls die von der abstrakten Basisklasse \textit{Importer} vorgebenen Methode zum Auslösen der Import-Routine. Auch hier wurde die Arbeit in drei Teilaufgaben gegliedert:
			\\\begin{itemize}	
				\item EventLoader - lädt die Daten vom Swiss Unihockey REST Service
				\item EventCreator - liefert Liste mit Event-Objekten
				\item EventSaver - Speichet die Events in der Datenbank ab\\
			\end{itemize}
	
			\noindent
			Da Events in der Regel nur einmal pro Saison importiert werden, müssen diese nicht auf Modifikationen überprüft werden.
	
	\subsection{MailingService}
		Um das Versenden von E-Mails zu vereinfachen, kapselt diese Klasse die Vorbereitung und Konfiguration eines \textit{SmtpClient} und einer \textit{MailMessage}. Dabei werden SMTP-Host und -Port, sowie die Credentials aus dem \textit{ConfigurationManager} gelesen.  

		\begin{lstlisting}[language=CSharp, caption=Verwendung des MailingService, label=lst:mailingservice, firstnumber=1]
	var mail = new MailingService();
	mail.SendMail("nle@hsr.ch", "nle@hsr.ch", "Test Subject", "Test Message");
	    \end{lstlisting}
    



\section{VoluntaryO.Web}
	Die komplette Implementierung des Frontends befindet sich im Projekt Voluntary.Web. Folgend wird beschrieben, wie VoluntaryO.Web aufgebaut ist und welche Konzepte verwendet wurden um die Darstellung des Frontends zu erreichen.

	\subsection{Projektaufbau}
		VoluntaryO.Web ist ein Projekt innerhalb der Solution. Der Projektaufbau wird in der Reihenfolge der Ordner beschrieben.
		\begin{figure}[h]
	  		\vspace{-5pt}
	    	\centering
	    	 \includegraphics[width=0.9\textwidth]{content/architekturdokumentation/images/web-1-Projektaufbau.png}
	  		\vspace{-25pt}
			\caption{Projektaufbau VoluntaryO.Web}
		\end{figure}
		
		\subsubsection{References}
			Das Projekt hat Referenzen auf VoluntaryO.Dal und VoluntarO.Service um auf die Datenpersistenz zugreifen zu können. Andere Referenzen ergeben sich automatisch durch den Einsatz von Entity Framework, und ASP.NET MVC.
		
		\subsubsection{AppStart}
			Hier werden wichtige Konfigurationen für ASP.NET vorgenommen. Besonders herauszuheben ist hier BundleConfig.cs. Unser Projekt bietet sechs verschiedene Bundles an. Jedes Bundle besteht aus mehreren CSS oder JS Files, die innerhalb eines Bundles in einem minimierten File an den Client ausgeliefert werden (Nur in Realese-Deploy). Dies bietet den Vorteil, dass während der Entwicklung der JS-Code gedebuggt werden kann und später eine möglichst hohe Effizien erreichbar ist. Ausserdem bietet Bundle-Config eine zentrale Managementlösung für JS- und CSS-Files.
			Im AppStart wird ausserdem die Konfiguration der Dependency Injection vorgenommen.
			Die Idee hinter Dependency Injection und allgemein Inversion-of-Control, ist die Anwendung des sogenannten Hollywood Prinzips: \"Don’t call us, we call you!\". 
			\\Als Dependency Injection Container wird \href{http://unity.codeplex.com/}{Unity} verwendet. Der Controller gibt in seinem Konstruktor an, welche Interfaces benötigt werden. \textit{Unity} wird diese Abhängigkeiten dann zur Laufzeit zur Verfügung stellen.
			\begin{lstlisting}[language=CSharp, caption=UnityConfig.cs, label=lst:unityconfig, firstnumber=1]
			container.RegisterType<IDataContextAsync, VoluntaryoContext>(new PerRequestLifetimeManager(), new InjectionConstructor())
		    .RegisterType<IUnitOfWork, UnitOfWork>(new PerRequestLifetimeManager())
		    .RegisterType<IUnitOfWorkAsync, UnitOfWork>(new PerRequestLifetimeManager())
		    .RegisterType<IEventRepository, EventRepository>()
		    .RegisterType<IHelperTaskRepository, HelperTaskRepository>()
		    .RegisterType<IRepository<HelperTaskType>, Repository<HelperTaskType>>()
		    .RegisterType<IMemberRepository, MemberRepository>()
		    .RegisterType<ITeamRepository, TeamRepository>()
		    .RegisterType<IRepositoryAsync<Skill>, Repository<Skill>>()
		    .RegisterType<IEventService, EventService>()
		    .RegisterType<IHelperTaskService, HelperTaskService>()
		    .RegisterType<ISkillService, SkillService>()
		    .RegisterType<IMemberService, MemberService>()
		    .RegisterType<ITeamService, TeamService>();
		    \end{lstlisting}
		    Der \textit{PerRequestLifetimeManager} hält die Instanz für den gesamten HTTP-Request. Somit erhält jeder Request ein eigener \textit{Context} und eine eigene \textit{Unit of Work}.
		    Zuletzt wird in AppStart noch die Konfiguration des Routings zum Web-API vorgenommen.

		\subsubsection{Content}
			Der Ordner Content enthält alle Bilder und CSS-Files, die für das Projekt benötigt werden. Diese sind unter anderem Bootstrap, Chosen und spezifische CSS –Files für VoluntaryO.

		\subsubsection{Controllers}
			Die Controller sind ein essentieller Teil des MVC-Patterns (Model-View-Controller), das wir durch ASP.NET verwenden.

			\begin{figure}[h]
		  		\vspace{-5pt}
		    	\centering
		    	 \includegraphics[width=0.9\textwidth]{content/architekturdokumentation/images/web-2-Controllers.png}
		  		\vspace{-25pt}
				\caption{PControllers}
			\end{figure}

			Sobald im Client eine URL mit folgendem Schema aufgerufen wird:
			/{controller}/{methode}
			Wird innerhalb des Controllers mit dem jeweiligen Name die jeweilige Methode aufgerufen. Für jede Methode innerhalb eines Controllers gibt es eine gleichnamige View im Ordner Views.
			In jeder Methode innerhalb des Controllers wird festgelegt, welche Rollen darauf zugreifen können.

			\begin{lstlisting}[language=CSharp, caption=EventController.cs, label=lst:Details firstnumber=1]
			// GET: /Event/Details/5
	        [Authorize(Roles = "Admin,Planer,Member")]
	        public ActionResult Details(int? id)
	        {
	            if (id == null)
	            {
	                return new HttpStatusCodeResult(HttpStatusCode.BadRequest);
	            }
	            var e = _eventService.Find((int) id);
	            if (e == null)
	            {
	                return HttpNotFound();
	            }
	            var model = new EventDetailViewModel(e, _teamService.GetAllTeams().ToList());
	            return View(model);
	        }
			\end{lstlisting}

			Speziell ist hier noch der Folder API. Er wird von verschiedenen Views benutzt um von der Client-Side über AJAX asynchrone Requests auf die Applikation auszuführen. Die Methoden sind hier nach den verschiedenen HTTP-Request-Codes benannt (GET, POST, DELETE usw.)

		\subsubsection{Scripts}
			Innerhalb des Folders Scripts sind alle JS-Files abgelegt. Grundsätzlich sind diese in Bootstrap, Chosen, FullCalender, JQuery und spezifische VoluntaryO-Scripts zu unterteilen. Die Scripts werden durch die zuvor beschriebenen Bundles in der View angezeigt.

		\subsubsection{View Models}

			\begin{figure}[h]
		  		\vspace{-5pt}
		    	\centering
		    	 \includegraphics[width=0.9\textwidth]{content/architekturdokumentation/images/web-3-ViewModels.png}
		  		\vspace{-25pt}
				\caption{ViewModels}
			\end{figure}

			Um die Websiten dynamisch darzustellen nutzen wir die ASP.NET Razor View Engine. Eine Razor-View ist immer mit einer Methode des Controllers verbunden. Das bedeutet auch, dass der Return-Wert der Methode sämtliche Daten mitliefern muss, die später in der View gebraucht werden. Da in einer View häufig mehr als nur ein standard Objekt aus dem Model gebraucht wird, haben viele Views ein Extra Viewmodel. Das bedeutet es wird extra eine Instanz eines speziell auf die View zugeschnittenen Objekt erstellt. Dieses View-Model enthält alle Referenzen auf andere Objekte, die innerhalb der View gebraucht werden. Durch den Einsatz des View-Model haben wir den Vorteil, dass pro View genau definiert ist, welche Daten wir brauchen. Der Zugriff auf das View-Model kann einfach getestet werden. Der Einsatz eines View-Models bietet aber auch einige Nachteile, beispielsweise werden manche Methoden redundant implementiert. Da wir die Implementation der Methoden aber sowieso auf die Service-Schicht abstrahiert haben, birgt dies für uns kein Problem. Da wir nicht durch das Model pro View eingeschränkt werden wollen bietet uns das View-Model den besten Dienst.

		\subsubsection{Views}

			\begin{figure}[h]
		  		\vspace{-5pt}
		    	\centering
		    	 \includegraphics[width=0.9\textwidth]{content/architekturdokumentation/images/web-4-Views.png}
		  		\vspace{-25pt}
				\caption{Views}
			\end{figure}

			Alle Controller besitzen ein entsprechendes Pendant im Ordner Views. Hier sind alle .cshtml Files abgelegt, die vom Client angezeigt werden können.
			Detailliert sieht man das innerhalb des Ordners Event. Für die Events gibt es eine Ansicht in Listenstruktur (Index), eine Detail-Ansicht (Details) und eine Erstell und Editier-Funktion (Create & Edit). Alle Files die mit einem Underscore beginnen (\_Events) sind Partial-Views die innerhalb anderer Views wiederverwendet werden können.

			In den Views ist auch der Javascript-Code zu finden, welcher die AJAX-Befehle ausführt. Ein Beispiel:

			\begin{lstlisting}[language=javascript, caption=text/javascript, label=lst:AJAX Example firstnumber=1]
			$(document).ready(function() {
	            $("#team-selection").chosen({
	                allow_single_deselect: true
	            });

	            $(".remove-helperTask").confirmation({
	                onConfirm: function(e, element) {
	                    console.log($(element));
	                    e.preventDefault();
	                    $.ajax({
	                        type: 'DELETE',
	                        url: '/api/helperTasks/' + $("#helperTaskId").val(),
	                        success: function(data) {
	                            getHelperTasks($('#calendar'));
	                            $("#taskdetailform")[0].reset();
	                        },
	                        error: function(data) {
	                            console.log(data);
	                        }
	                    });
	                }
	            });
	        });
			\end{lstlisting}

	\subsection{Controllers}
		Folgend werden die vier wichtigsten Controller und der REST-API-Controller detailliert beschrieben
		\subsubsction{HomeController}
		asdf
		\subsubsction{EventController}
		adsf
		\subsubsction{HelperTaskController}
		asd
		\subsubsction{AccountController}
		asd
		\subsubsction{ApiController}
		asdf




	\subsection{Model-View-Controller}
		Für die Presentationschicht setzen wir MVC ein, da ASP.NET eine gute Unterstützung für dieses Modell bietet. Auf alle Seiten wird mittels Controller zugegriffen. Die Views werden mithilfe der Razorsprache erstellt.

	\subsection{View Model und Razor}
		Um die Websiten dynamisch darzustellen nutzen wir die ASP.NET Razor View Engine. Eine Razor-View ist immer mit einer Methode des Controllers verbunden. Das bedeutet auch, dass der Return-Wert der Methode sämtliche Daten mitliefern muss, die später in der View gebraucht werden. Da in einer View häufig mehr als nur ein standard Objekt aus dem Model gebraucht wird, haben viele Views ein Extra Viewmodel. Das bedeutet es wird extra eine Instanz eines speziell auf die View zugeschnittenen Objekt erstellt. Dieses View-Model enthält alle Referenzen auf andere Objekte, die innerhalb der View gebraucht werden.
		Durch den Einsatz des View-Model haben wir den Vorteil, dass pro View genau definiert ist, welche Daten wir brauchen. Der Zugriff auf das View-Model kann einfach getestet werden.
		Der Einsatz eines View-Models bietet aber auch einige Nachteile, beispielsweise werden manche Methoden redundant implementiert. Da wir die Implementation der Methoden aber sowieso auf die Service-Schicht abstrahiert haben, birgt dies für uns kein Problem.
		Da wir nicht durch das Model pro View eingeschränkt werden wollen bietet uns das View-Model den besten Dienst.

	\subsection{ASP.NET Identity (Authentifizierung und Authentisierung)}
		Wir verwenden ASP.NET Identity in Version 2.0 für die Authentifizierung und Authentisierung. Folgende benutzerrelevanten Funktionen werden vom Projekt unterstützt:
		\\\begin{itemize}	
			\item Login / Logout
			\item Passwort vergessen
			\item Benutzer anlegen, importieren, bearbeiten und löschen (Admin)
			\item Rollenzuweisung
		\end{itemize}
		Diese Funktionalitäten sind im \textit{AccountController.cs} abgehandelt. Die Logik für die obigen Funktionen sind in der Klasse \textit{UserManager.cs}, welche von \textit{ASP.Net Identity} implementiert ist. Die benutzerrelevanten Funktionen sind daher Indirektionen auf den \textit{UserManager}.

		\subsubsection{Rollen}
			\begin{itemize}
				\item Member
				\item Planer
				\item Admin
			\end{itemize}
			Die Rollen werden für die Authentisierung für Controller-Actions und Views benötigt.
			\begin{lstlisting}[language=CSharp, caption=AccountController.cs, label=lst:accountcontroller, firstnumber=1]
// GET: /Acount/Edit/madminis
[Authorize(Roles = "Admin")]
public ActionResult Edit(string id, ManageMessageId? Message = null) {}
			\end{lstlisting}

			Authentisierung in View:
			\begin{lstlisting}[language=CSharp, caption=\_Layout.cshtml, label=lst:layoutauthentisierung, firstnumber=1]
@if (User.IsInRole("Admin") || User.IsInRole("Admin"))
			\end{lstlisting}

		\subsubsection{Passwortverschlüsselung}
		Das Passwort wird als Hash (\textit{HMACSHA256}) in der Datenbank abgespeichert. Das Projekt verwendet die Standardimplementation von Asp.Net Idenity mit \textit{PasswordHasher}. Der \textit{PasswordHasher} verwendet die \href{http://de.wikipedia.org/wiki/PBKDF2}{\textit{PBKDF2}} um den Hash inkl. einen Saltwert zu strecken. Die Streckung wird 5000 mal durchgeführt. Diese Verkettung erschwert es, per Brute-Force-Methode aus dem Schlüssel auf das ursprüngliche Passwort zu schliessen.


	\subsection{Bootstrap, CSS und JS}
		Das Front-End Framework, dass wir einsetzen ist Twitter Bootstrap. Zusätzlich verwenden wir ein verändertes CSS-File um nicht den typischen Bootstrap-Look zu erhalten.
		Zusätzlich setzen wir die Javascript-Library JQuery ein.
		ASP.NET bietet einen für die Einbindung von CSS- und JS-Files einen Bundle-Konfigurator an. So können CSS-Files oder JS-Files in Bundles gruppiert werden. Per View kann man dann genau definieren welche Bundles aktuell gebraucht werden. ASP.NET bindet einem dann automatisch die entsprechen Files in die HTML-Datei ein.
		Besonders nützlich ist hier, dass während der Entwicklungs-Phase die Files noch einzeln und noch nicht minifiziert übertragen werden. Sobald man die Applikation aber auf den Server deployed werden die Files automatisch in ein File gerendert und auch minifiziert, was für die Übertragungsgeschwindigkeit wichtig sein kann.
		Dabei wird pro Bundle ein File ausgeliefert, was besonders für das HTML-Caching wichtig ist.

	\subsection{AJAX und WebAPI}
		Unser Ziel ist es, dass unsere Applikation möglichst dynamisch sein soll. Dafür ist der Einsatz von asynchronen Request extrem wichtig. In unserer Applikationen sind mehrere verschiedene Implementationsarten dieses Konzeptes eingesetzt worden. Wenn möglich wurden die direkt von ASP.NET unterstützten Features genutzt. Hier wird über ein sogenanntes Ajax-Form in der Razor-Syntax ein AJAX Befehl konfiguriert. Als Return-Value wird hier ein Partial-View zurückgegeben. Dieser ersetzt dann ein bestehendes Partial-View.
		Besonders gut ist hier, dass unobstrusive-AJAX eingesetzt wird. Das bedeutet dieser Code würde auch funktionieren, wenn Javascript nicht aktiviert wäre.
		Für die komplexeren AJAX-Requests, die nicht mehr durch die ASP.NET Boardmittel gelöst werden können, haben wir ein REST Web-Api implementiert. Dieses API ermöglicht es durch AJAX-Request JSON Daten zurückzukriegen und diese dann dynamisch auf der Seite zu implementieren.
		Das Web-Api ist massgebend für die dynamik der Page.
