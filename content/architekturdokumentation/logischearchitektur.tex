%!TEX root = ../../architekturdokumentation.tex
\chapter{Logische Architektur}
\section{3-Tier-Architektur}
	Um eine möglichst hohe Abstraktion unseres Codes zu erhalten, haben wir uns für eine 3-Tier-Architektur entschieden, die unsere Applikation in die drei Schichten Web, Service und DataAccesLayer (Dal) unterteilt. Alle Datenbankspezifischen Operationen sollen im DAL vorgenommen werden. Die Zugriffe auf externe Schnittstellen über den Service-Layer. Alle Zugriffe aus dem Web sollen über die Web-Layer behandelt werden. 
    \begin{figure}[H]
    	\centering
		\includegraphics[width=0.4\textwidth]{content/architekturdokumentation/images/LogischeArchitektur_plain.png}
  		\vspace{-10pt}
    	\caption{Logische Architektur}
	\end{figure}

\newpage
\section{VoluntaryO.Dal (Data Access Layer)}
    \begin{figure}[h]
  		\vspace{-5pt}
    	\centering
    	\includegraphics[width=0.9\textwidth]{content/architekturdokumentation/images/VoluntaryO_Dal_Overview.png}
  		\vspace{-20pt}
    	\caption{Klassen in VoluntaryO.Dal}
	\end{figure}
	Jede Model-Klasse implementiert die Schnittstelle \textit{IObjectState}, resp. leitet von der abstrakten Klasse \textit{Entity} ab. Die Verwendung von ASP.NET Identity ist im Kapitel VoluntaryO.Web näher beschrieben.
	
	\subsection{Vergleich für Objekte}
		Einzig für die Member-Klasse wurde die Vergleichsmethode (Equals) überschrieben. Folgende Attribute der Klasse sind relevant:
		\\\begin{itemize}
			\item \textit{UserName}
			\item \textit{Firstname}
			\item \textit{Lastname}
			\item \textit{Birthdate}
		\end{itemize}
		Der Vergleich wurde nur für die Mitgliedsimport-Funktion überschrieben. Für die übrigen Model-Klassen wird die standard Vergleichsmethode beibehalten (Vergleich der Referenz bei Referenztypen).

	\subsection{Repository / Unit of Work Pattern}
		Durch die Verwendung des \href{https://genericunitofworkandrepositories.codeplex.com/}{\textit{Generic Unit of Work \& Repositories Framework}} (in "'Framework"' abgelegt) ergeben sich folgende Vorteile:
		\\\begin{itemize}	
			\item Austauschbarkeit ORM
			\item Testbarkeit
			\item Jeder Request hat eigene Unit of Work (siehe später Dependecy Injection)
			\item Reduktion der Datenbankabfragen, da nur noch über Unit of Work commited wird
			\item Generische Abfragen von Entitäten
		\end{itemize}
		Folgende Konventionen ergeben sich aus für das Projekt:
		\\\begin{itemize}
			\item Jede Entität implementiert \textit{IObjectState}
			\item Repository kann über \textit{partial Classes} erweitern werden
			\item Oder Repository Klassen implementieren \textit{IRepository} und erben von \textit{Repository}
		\end{itemize}

	\subsubsection{Konventionen für eigene Repository Erweiterungen}
		\begin{itemize}
			\item Objekt(-e) abrufen/suchen \textit{Find*}
			\item Objekt(-e) einfügen \textit{Insert*}
			\item Objekt ändern \textit{Update*}
			\item Objekt hinzufügen \textit{Delete*}
			\item Zuweisung zweier Objekte hinzufügen \textit{Assign*}
			\item Zuweisung zweier Objekte entfernen \textit{Unassign*}
		\end{itemize}

	\subsection{Unit Tests mit Effort}
		\href{https://effort.codeplex.com/}{\textit{Effort}} ist ein ADO.Net Provider und erstellt jeweils eine In-Memory Datenbank für das EF. Die temporäre Datenbank ermöglicht uns ein Testen des Codes, ohne Datenbankzugriffe zu mocken.
		\begin{lstlisting}[language=CSharp, caption=Verwendung Effort für Unit Tests in EffortTest.cs, label=lst:effortunittest, firstnumber=1]
			DbConnection Connection = DbConnectionFactory.CreateTransient();
			VoluntaryoContext VoluntaryoContext = new VoluntaryoContext(_Connection);
	    \end{lstlisting}
	    Den einzelnen Unit Tests steht so eine saubere und testbare Datenbank zur Verfügung. Jeder UnitTest erbt von der Abstrakten Klasse \textit{EffortTest}.

	\subsection{Relationales Modell}
	    \begin{figure}[h]
	  		\vspace{-5pt}
	    	\centering
	    	\includegraphics[width=0.7\textwidth]{content/architekturdokumentation/images/edmx.png}
	  		\vspace{-20pt}
	    	\caption{Relationales Modell für VoluntaryO.Dal}
		\end{figure}
		\subsubsection{Mapping}
			Das Mapping wird mit dem \textit{modelBuilder} des EF gesteuert. Bspw.
			\begin{lstlisting}[language=CSharp, caption=Mapping in VoluntaryoContext.cs, label=lst:mappingcontextcs, firstnumber=1]
	// TeamEventMapping
	modelBuilder.Entity<Team>()
	    .HasMany(t => t.Events)
	    .WithMany(e => e.Teams)
	    .Map(mc =>
	    {
	        mc.MapLeftKey("TeamID");
	        mc.MapRightKey("EventID");
	        mc.ToTable("TeamEventMappings");
	    });
		    \end{lstlisting}

\newpage
\section{VoluntaryO.Service}
	Dieses Subprojekt dient als Abstraktionsschicht zwischen den Packages aus dem DAL und deren Verwendung im Web Layer und enthält die eigentliche Business-Logik. Dadurch wird die Logik in den Actions der Controller reduziert und eine höhere Kohäsion erreicht. Die Aufteilung  von Namespaces und Ordner wurde aufgrund der Zugehörigkeit zu den Entitätsklassen im Domainmodell gewählt. Aufgabenspezifische Service-Komponenten wurden ebenfalls in der obersten Hierarchiestufe des Service-Projekts in einem separaten Ordner untergebracht.
	
	\begin{figure}[H]
	    	\centering
	    	 \includegraphics[width=0.9\textwidth]{content/architekturdokumentation/images/Service_Overview.png}
	  		\vspace{-25pt}
			\caption{Packageübersicht des Service-Layers und Service Pattern}
	\end{figure}
	
	\subsection{Verwendung der Service-Klassen}
	Für jedes Repository existiert eine zugehörige Service-Klasse, welche das \textit{IService} Interface implementiert. Dadurch wird die Kopplung zwischen Respositories und Controllern reduziert und eine einheitliche Schnittstelle zur Business-Logik geschaffen:
	
	\begin{figure}[H]
	    	\centering
	    	 \includegraphics[width=0.9\textwidth]{content/architekturdokumentation/images/Service_Framework_Code.png}
	  		\vspace{-15pt}
			\caption{Aufrufhierarchie über Layers anhand eines Beispiels}
	\end{figure}
	
	\subsection{Aufbau der Service-Klassen}
	Die Vererbungshierarchie aller Service-Klassen (Ausnahme: MailingService) folgt immer dem selben Schema. Dadurch kann ein einheitliches Design verwendet werden und es ist klar, wo Erweiterungen integriert werden können.
	
	\begin{figure}[H]
	    	\centering
	    	 \includegraphics[width=0.9\textwidth]{content/architekturdokumentation/images/Service_Class_Hierarchy.png}
	  		\vspace{-15pt}
			\caption{Referenz-Hierarchie für Service-Klassen}
	\end{figure}
	
	Pro Repository wird ein Interface definiert (\textit{IDemoService}), welches alle spezifischen Methoden vorgibt. Jede Implementierung dieser Schnittstelle (\textit{DemoService, AnotherDemoService}) erbt zusätzlich von \textit{Service}. Der Dependency Injection Container (Unity) ist verantwortlich für die Instanziierung der richtigen Implementierung.
	
	
	\subsection{Service-Klassen im Überblick}
	

	
	\subsubsection{MemberService}
	
	Die MemberService-Klasse funktioniert als Schnittstelle zum MemberRepository. Speziell hervorzuheben ist die \textit{ImportMember}-Methode. Sie erlaubt es, freistehende Member-Objekte (externe Members), die nicht aus dem Repository stammen, kollisionsfrei zu integrieren.
	
	\begin{table}[H]
        \tablestyle
        \tablealtcolored
        \begin{tabularx}{\textwidth}{X X}
        \tableheadcolor
            \tablehead Methode & 
            \tablehead Beschreibung \\  
        \tablebody
            bool CheckIfMemberExists(Member member) & 
            Überprüft, ob ein Mitglied mit selbem Vornamen, Nachnamen und Geburtsdatum existiert. Falls nicht, wird davon ausgegangen, dass diese Person noch nicht im System vorhanden ist.  \tabularnewline
            
            bool CheckIfMemberWasModified(Member member) &
            Überprüft, ob das Mitglied existiert und Kontaktdaten oder Teamzugehörigkeit gewechselt haben. \tabularnewline
            
			Boolean ImportMember(Member member) &
			Versucht externe Members im System zu speichern. Dabei wird unterschieden, ob es sich um ein neues Mitglied, ein bestehendes mit abweichenden Angaben oder ein bestehendes, unverändertets Mitglied handelt. \tabularnewline      
			
			bool ImportMember(Member member, bool overwriteIfModified) &
			Importiert externe Members und überschreibt bestehende Member-Objekte, falls das \textit{overwriteIfModified}-Flag gesetzt wurde. \tabularnewline
            
			void UpdateMember(Member member)  &
			Aktualisiert Kontaktdaten und Teamzugehörigkeit eines Mitglieds. \tabularnewline
			
			void AssignMemberAndTeamWithInsert(Member member, string teamName) &
			Weist dem Mitglied ein echtes Team-Objekt aus dem Repository zu oder erstellt ein solches anhand des übergebenen Strings, falls es noch nicht existiert. \tabularnewline
			
			ICollection<Member> UnmodifiedRecords() &
			Externe Members, die beim Importieren übersprungen wurden, weil sie nicht verändert wurden. \tabularnewline
			
			ICollection<Member> ModifiedRecords() &
			Externe Members, die verändert und deshalb nicht abgespeichert wurden. \tabularnewline
			
			ICollection<Member> NewRecords() &
			Externe Members, die zuvor noch nicht im System vorhanden waren und abgespeichert wurden. \tabularnewline
			
				
			         
        \tableend
        
        \end{tabularx} 
    \end{table}
	
	
	\subsection{EventService}
		Der EventService verbindet den EventController mit dem Event- und HelperTaskRepository. Die Logik zum Verwalten der Relationen zwischen Helfereinsätzen, Events Mannschaften wurde hier implementiert.
		
		\begin{table}[H]
        \tablestyle
        \tablealtcolored
        \begin{tabularx}{\textwidth}{X X}
        \tableheadcolor
            \tablehead Methode & 
            \tablehead Beschreibung \\  
        \tablebody
            void AssignEventAndTeam(int eventId, int teamId) & 
            Fragt Event und Team anhand von Id's aus den Repositories ab und stellt die Relation her.  \tabularnewline
            
            void UnassignEventAndTeam(int eventId, int teamId) & 
            Fragt Event und Team anhand von Id's aus den Repositories ab und hebt die Relation auf.  \tabularnewline
            
           void InsertHelperTask(HelperTask task, int taskType, int eventId) &
           Selektiert Event- und HelperTask-Objekte, um dem Event einen Helfereinsatz zuzuordnen. \tabularnewline
           
           void DeleteHelperTask(int helperTaskId) &
           Löscht den Helfereinsatz anhand der Id. Das Event wird dazu nicht benötigt. \tabularnewline
           
           void UnassignEventAndAllTeams(Event e) &
           Entfernt alle Teams eines Events. \tabularnewline
           
           void	UnassignEventAndAllHelperTasks(Event e) &
           Entfernt alle Helfereinsätze eines Events. \tabularnewline
           
           void AssignEventAndTeam(Event e, Team team) &
           Ordnet dem Event ein weiteres zuständiges Team zu. \tabularnewline
           
           void AssignEventAndHelperTask(Event e, int helperTaskId) &
           Ordnet dem Event einen weiteren Helfereinsatz zu. \tabularnewline
           
           bool Contains(Event e) &
           Überprüft, ob ein Event bereits im Repository existiert. \tabularnewline
			
				
			         
        \tableend
        
        \end{tabularx} 
    \end{table}
	
	\subsection{TeamService}
		Bietet Zugriff auf die Methoden des TeamRepository und stellt Erweiterungspunkt für komplexere Logik dar. Diese Klasse wird im \textit{EventController} verwendet.
		
		\begin{table}[H]
        \tablestyle
        \tablealtcolored
        \begin{tabularx}{\textwidth}{X X}
        \tableheadcolor
            \tablehead Methode & 
            \tablehead Beschreibung \\  
        \tablebody
            IEnumerable<Team> GetAllTeams() & 
            Selektiert alle Teams aus dem Repository.  \tabularnewline
            
            Team GetTeamById(int id) &
            Selektiert Team aus Repository anhand der Id. \tabularnewline
	
			         
        \tableend
        
        \end{tabularx} 
    \end{table}
		
	
	\subsection{SkillService}
		Bietet Zugriff auf die Methoden des SkillRepository und stellt Erweiterungspunkt für komplexere Logik dar.
		
		\begin{table}[H]
        \tablestyle
        \tablealtcolored
        \begin{tabularx}{\textwidth}{X X}
        \tableheadcolor
            \tablehead Methode & 
            \tablehead Beschreibung \\  
        \tablebody
            public IEnumerable<Skill> GetAllSkills() & 
            Selektiert alle Skills aus dem Repository.  \tabularnewline
	
			         
        \tableend
        
        \end{tabularx} 
    \end{table}
	
	\subsection{HelperTaskService}
		Die HelferTaskService-Klasse wird von mehreren Controllern verwendet. Sie enthält wichtige Business-Logik zum An- und Abmelden für einen Helfereinsatz und prüft dazu folgende Bedingungen:
		\\\begin{itemize}
			\item Helfereinsatz ist noch nicht mit ausreichend vielen Mitgliedern belegt
			\item Mitglied hat sich noch nicht für diesen Helfereinsatz angemeldet
			\item Keine Zeitüberschneidung mit anderen Helfereinsätzen vorhanden
			\item Fähigkeiten für den Einsatz sind gegeben
		\end{itemize}
		
		\begin{table}[H]
        \tablestyle
        \tablealtcolored
        \begin{tabularx}{\textwidth}{X X}
        \tableheadcolor
            \tablehead Methode & 
            \tablehead Beschreibung \\  
        \tablebody
           void SubmitForTask(Member member, HelperTask task) & 
           Versucht, das Mitglied für einen Helfereinsatz zu registrieren. Dazu müssen alle oben genannten Bedinungen erfüllt sein. \tabularnewline
           
           void RemoveFromTask(Member member, HelperTask task) &
           Meldet das Mitglied wieder vom Helfereinsatz ab. \tabularnewline
           
           void UnassignAllMembers(HelperTask helperTask) &
           Meldet alle Mitglieder eines Helfereinsatzes ab. \tabularnewline
           
           
	
		\tableend
        
        \end{tabularx} 
    \end{table}
	
	\subsection{Imports}
		Für das Importieren von Mitgliedern und Events wird jeweils eine eigene Serviceklasse verwendet. Die abstrakte Klasse \textit{Importer} dient dazu als Basis und schreibt die Methode \textit{executeImport} vor.
		
	    \begin{figure}[h]
	  		\vspace{-5pt}
	    	\centering
	    	 \includegraphics[width=0.9\textwidth]{content/architekturdokumentation/images/ImportPackageDesign.png}
	  		\vspace{-25pt}
			\caption{Importer-Klassen und deren direkte Collaborators}
		\end{figure}
		
		\subsubsection{MemberImporter}
			Der MemberImporter implementiert die Methode \textit{executeImport}, welche das importieren von Mitgliedern startet. Um eine höhere Kohäsion und niedrige Kopplung zu erreichen, wurden die Teilaufgaben ausgelagert:
			\\\begin{itemize}	
				\item MemberLoader - benutzt die webling.ch Schnittstelle und lädt das CSV
				\item MemberParser - liest den CSV-String und speichert die Werte in eine Liste von Hashtables
				\item MemberMapper - gibt Member-Objekte an MemberService zum Speichern weiter\\
			\end{itemize}



		\subsubsection{EventImporter}
			Der EventImporter implementiert ebenfalls die von der abstrakten Basisklasse \textit{Importer} vorgebenen Methode zum Auslösen der Import-Routine. Auch hier wurde die Arbeit in drei Teilaufgaben gegliedert:
			\\\begin{itemize}	
				\item EventLoader - lädt die Daten vom Swiss Unihockey REST Service
				\item EventCreator - liefert Liste mit Event-Objekten
				\item EventSaver - Speichet die Events in der Datenbank ab\\
			\end{itemize}
	
			\noindent
			Da Events in der Regel nur einmal pro Saison importiert werden, müssen diese nicht auf Modifikationen überprüft werden.
	
	\subsection{MailingService}
		Um das Versenden von E-Mails zu vereinfachen, kapselt diese Klasse die Vorbereitung und Konfiguration eines \textit{SmtpClient} und einer \textit{MailMessage}. Dabei werden SMTP-Host und -Port, sowie die Credentials aus dem \textit{ConfigurationManager} gelesen.  

		\begin{lstlisting}[language=CSharp, caption=Verwendung des MailingService, label=lst:mailingservice, firstnumber=1]
	var mail = new MailingService();
	mail.SendMail("nle@hsr.ch", "nle@hsr.ch", "Test Subject", "Test Message");
	    \end{lstlisting}
    



\section{VoluntaryO.Web}
	Die komplette Implementierung des Frontends befindet sich im Projekt Voluntary.Web. Folgend wird beschrieben, wie VoluntaryO.Web aufgebaut ist und welche Konzepte verwendet wurden um die Darstellung des Frontends zu erreichen.

	\subsection{Projektaufbau}
		VoluntaryO.Web ist ein Projekt innerhalb der Solution. Der Projektaufbau wird in der Reihenfolge der Ordner beschrieben.

		\begin{figure}[H]
	    	\centering
	    	 \includegraphics[width=0.3\textwidth]{content/architekturdokumentation/images/web-1-Projektaufbau.png}
	  		\vspace{-10pt}
			\caption{Projektaufbau VoluntaryO.Web}
		\end{figure}

		\subsubsection{References}
			Das Projekt hat Referenzen auf VoluntaryO.Dal und VoluntarO.Service um auf die Datenpersistenz zugreifen zu können. Andere Referenzen ergeben sich automatisch durch den Einsatz von Entity Framework, und ASP.NET MVC.
		
		\subsubsection{AppStart}
			Hier werden wichtige Konfigurationen für ASP.NET vorgenommen. Besonders herauszuheben ist hier BundleConfig.cs. Unser Projekt bietet sechs verschiedene Bundles an. Jedes Bundle besteht aus mehreren CSS oder JS Files, die innerhalb eines Bundles in einem minimierten File an den Client ausgeliefert werden (Nur in Realese-Deploy). Dies bietet den Vorteil, dass während der Entwicklung der JS-Code gedebuggt werden kann und später eine möglichst hohe Effizien erreichbar ist. Ausserdem bietet Bundle-Config eine zentrale Managementlösung für JS- und CSS-Files.
			Im AppStart wird ausserdem die Konfiguration der Dependency Injection vorgenommen.
			Die Idee hinter Dependency Injection und allgemein Inversion-of-Control, ist die Anwendung des sogenannten Hollywood Prinzips: \"Don’t call us, we call you!\". 
			\\Als Dependency Injection Container wird \href{http://unity.codeplex.com/}{Unity} verwendet. Der Controller gibt in seinem Konstruktor an, welche Interfaces benötigt werden. \textit{Unity} wird diese Abhängigkeiten dann zur Laufzeit zur Verfügung stellen.
			\begin{lstlisting}[language=CSharp, caption=UnityConfig.cs, label=lst:unityconfig, firstnumber=1]
			container.RegisterType<IDataContextAsync, VoluntaryoContext>(new PerRequestLifetimeManager(), new InjectionConstructor())
		    .RegisterType<IUnitOfWork, UnitOfWork>(new PerRequestLifetimeManager())
		    .RegisterType<IUnitOfWorkAsync, UnitOfWork>(new PerRequestLifetimeManager())
		    .RegisterType<IEventRepository, EventRepository>()
		    .RegisterType<IHelperTaskRepository, HelperTaskRepository>()
		    .RegisterType<IRepository<HelperTaskType>, Repository<HelperTaskType>>()
		    .RegisterType<IMemberRepository, MemberRepository>()
		    .RegisterType<ITeamRepository, TeamRepository>()
		    .RegisterType<IRepositoryAsync<Skill>, Repository<Skill>>()
		    .RegisterType<IEventService, EventService>()
		    .RegisterType<IHelperTaskService, HelperTaskService>()
		    .RegisterType<ISkillService, SkillService>()
		    .RegisterType<IMemberService, MemberService>()
		    .RegisterType<ITeamService, TeamService>();
		    \end{lstlisting}
		    Der \textit{PerRequestLifetimeManager} hält die Instanz für den gesamten HTTP-Request. Somit erhält jeder Request ein eigener \textit{Context} und eine eigene \textit{Unit of Work}.
		    Zuletzt wird in AppStart noch die Konfiguration des Routings zum Web-API vorgenommen.

		\subsubsection{Content}
			Der Ordner Content enthält alle Bilder und CSS-Files, die für das Projekt benötigt werden. Diese sind unter anderem Bootstrap, Chosen und spezifische CSS –Files für VoluntaryO.

		\subsubsection{Controllers}
			Die Controller sind ein essentieller Teil des MVC-Patterns (Model-View-Controller), das wir durch ASP.NET verwenden.

			\begin{figure}[H]
		    	\centering
		    	 \includegraphics[width=0.5\textwidth]{content/architekturdokumentation/images/web-2-Controllers.png}
		  		\vspace{-5pt}
				\caption{Controllers}
			\end{figure}

			Sobald im Client eine URL mit folgendem Schema aufgerufen wird:
			/{controller}/{methode}
			Wird innerhalb des Controllers mit dem jeweiligen Name die jeweilige Methode aufgerufen. Für jede Methode innerhalb eines Controllers gibt es eine gleichnamige View im Ordner Views.
			In jeder Methode innerhalb des Controllers wird festgelegt, welche Rollen darauf zugreifen können.

			\begin{lstlisting}[language=CSharp, caption=EventController.cs, label=lst:Details firstnumber=1]
			// GET: /Event/Details/5
	        [Authorize(Roles = "Admin,Planer,Member")]
	        public ActionResult Details(int? id)
	        {
	            if (id == null)
	            {
	                return new HttpStatusCodeResult(HttpStatusCode.BadRequest);
	            }
	            var e = _eventService.Find((int) id);
	            if (e == null)
	            {
	                return HttpNotFound();
	            }
	            var model = new EventDetailViewModel(e, _teamService.GetAllTeams().ToList());
	            return View(model);
	        }
			\end{lstlisting}

			Speziell ist hier noch der Folder API. Er wird von verschiedenen Views benutzt um von der Client-Side über AJAX asynchrone Requests auf die Applikation auszuführen. Die Methoden sind hier nach den verschiedenen HTTP-Request-Codes benannt (GET, POST, DELETE usw.)

		\subsubsection{Scripts}
			Innerhalb des Folders Scripts sind alle JS-Files abgelegt. Grundsätzlich sind diese in Bootstrap, Chosen, FullCalender, JQuery und spezifische VoluntaryO-Scripts zu unterteilen. Die Scripts werden durch die zuvor beschriebenen Bundles in der View angezeigt.

		\subsubsection{View Models}

			\begin{figure}[H]
		    	\centering
		    	 \includegraphics[width=0.5\textwidth]{content/architekturdokumentation/images/web-3-ViewModels.png}
				\caption{ViewModels}
			\end{figure}

			Um die Websiten dynamisch darzustellen nutzen wir die ASP.NET Razor View Engine. Eine Razor-View ist immer mit einer Methode des Controllers verbunden. Das bedeutet auch, dass der Return-Wert der Methode sämtliche Daten mitliefern muss, die später in der View gebraucht werden. Da in einer View häufig mehr als nur ein standard Objekt aus dem Model gebraucht wird, haben viele Views ein Extra Viewmodel. Das bedeutet es wird extra eine Instanz eines speziell auf die View zugeschnittenen Objekt erstellt. Dieses View-Model enthält alle Referenzen auf andere Objekte, die innerhalb der View gebraucht werden. Durch den Einsatz des View-Model haben wir den Vorteil, dass pro View genau definiert ist, welche Daten wir brauchen. Der Zugriff auf das View-Model kann einfach getestet werden. Der Einsatz eines View-Models bietet aber auch einige Nachteile, beispielsweise werden manche Methoden redundant implementiert. Da wir die Implementation der Methoden aber sowieso auf die Service-Schicht abstrahiert haben, birgt dies für uns kein Problem. Da wir nicht durch das Model pro View eingeschränkt werden wollen bietet uns das View-Model den besten Dienst.

		\subsubsection{Views}

			\begin{figure}[H]
		    	\centering
		    	 \includegraphics[width=0.3\textwidth]{content/architekturdokumentation/images/web-4-Views.png}
		  		\vspace{-10pt}
				\caption{Views}
			\end{figure}

			Alle Controller besitzen ein entsprechendes Pendant im Ordner Views. Hier sind alle .cshtml Files abgelegt, die vom Client angezeigt werden können.
			Detailliert sieht man das innerhalb des Ordners Event. Für die Events gibt es eine Ansicht in Listenstruktur (Index), eine Detail-Ansicht (Details) und eine Erstell und Editier-Funktion (Create \& Edit). Alle Files die mit einem Underscore beginnen (\_Events) sind Partial-Views die innerhalb anderer Views wiederverwendet werden können.

			In den Views ist auch der Javascript-Code zu finden, welcher die AJAX-Befehle ausführt. Ein Beispiel:

			\begin{lstlisting}[language=javascript, caption=text/javascript, label=lst:AJAX Example firstnumber=1]
			$(document).ready(function() {
	            $("#team-selection").chosen({
	                allow_single_deselect: true
	            });

	            $(".remove-helperTask").confirmation({
	                onConfirm: function(e, element) {
	                    console.log($(element));
	                    e.preventDefault();
	                    $.ajax({
	                        type: 'DELETE',
	                        url: '/api/helperTasks/' + $("#helperTaskId").val(),
	                        success: function(data) {
	                            getHelperTasks($('#calendar'));
	                            $("#taskdetailform")[0].reset();
	                        },
	                        error: function(data) {
	                            console.log(data);
	                        }
	                    });
	                }
	            });
	        });
			\end{lstlisting}

	\subsection{Controllers}
		Folgend werden die vier wichtigsten Controller und der REST-API-Controller detailliert beschrieben
		\subsubsection{HomeController}
			
			\begin{table}[H]
		        \tablestyle
		        \tablealtcolored
		        \begin{tabularx}{\textwidth}{p{3cm} p{0.8cm} p{6cm} X p{1.2cm}}
		        \tableheadcolor
		            \tablehead Adresse & 
		            \tablehead HTTP &
		            \tablehead Beschreibung &
		            \tablehead Parameter &
		            \tablehead Rechte \\  
		        \tablebody
		        	/Home/Index &
		        	GET &
		        	Gibt alle offenen Helfereinsätze eines Members zurück &
		        	- &
		        	Admin, Planer, Member
		        	\tabularnewline
		        \tableend
		        \end{tabularx} 
		    \end{table}

		\subsubsection{EventController}

			\begin{table}[H]
		        \tablestyle
		        \tablealtcolored
		        \begin{tabularx}{\textwidth}{p{3cm} p{0.8cm} p{6cm} X p{1.2cm}}
		        \tableheadcolor
		            \tablehead Adresse & 
		            \tablehead HTTP &
		            \tablehead Beschreibung &
		            \tablehead Parameter &
		            \tablehead Rechte \\  
		        \tablebody
		        	/Event/Index &
		        	GET &
		        	Gibt alle Events zurück. Möglichkeit für Filterung und Sortierung. Falls über AJAX angefragt wird Partial View zurückggeben. &
		        	string sortOrder
					string searchString &
		        	Admin, Planer, Member
		        	\tabularnewline

		        	/Event/Details/{id} &
		        	GET &
		        	Gibt details eines Events als EventDetailViewModel zurück &
		        	int? Id &
		        	Admin, Planer, Member
		        	\tabularnewline

		        	/Event/DeleteTeam &
		        	GET &
		        	Entfernt Teamzuweisung aus einem spezifischen Event &
		        	int teamId, int eventId &
		        	Planer
		        	\tabularnewline

		        	/Event/AddTeam &
		        	GET &
		        	Fügt Teamzuweisung zu einem spezifischen Event hinzu &
		        	int teamId, int eventId &
		        	Planer
		        	\tabularnewline

		        	/Event/Edit/{id} &
		        	GET &
		        	Gibt zu bearbeitenden Event zurück. Zusätzlich eine Liste mit allen vorhandenen Teams um in View Auswahlliste anzubieten. Return als EditEventViewModel &
		        	int? Id &
		        	Planer
		        	\tabularnewline

		        	/Event/Edit/ &
		        	POST &
		        	Speichert die Änderungen die an Event vorgenommen wurden in DB. Falls ModelState nicht Valid ist, wird Edit-View erneut angezeigt. &
		        	EditEventViewModel model &
		        	Planer
		        	\tabularnewline

		        	/Event/Create &
		        	GET &
		        	Zeigt Create View &
		        	- &
		        	Planer
		        	\tabularnewline

		        	/Event/Create &
		        	POST &
		        	Erstellt Event mit den angegeben Daten &
		        	Event @event &
		        	Planer
		        	\tabularnewline

		        \tableend
		        \end{tabularx} 
		    \end{table}

		\subsubsection{HelperTaskController}
		
		\begin{table}[H]
		        \tablestyle
		        \tablealtcolored
		        \begin{tabularx}{\textwidth}{p{3cm} p{0.8cm} p{6cm} X p{1.2cm}}
		        \tableheadcolor
		            \tablehead Adresse & 
		            \tablehead HTTP &
		            \tablehead Beschreibung &
		            \tablehead Parameter &
		            \tablehead Rechte \\  
		        \tablebody
		        	/HelperTask/Index &
		        	GET &
		        	Gibt alle noch bevorstehenden und abgeschlossene Helfereinsätze eines Members zurück &
		        	- &
		        	Admin, Planer, Member
		        	\tabularnewline

		        	/HelperTask/Create &
		        	GET &
		        	Gibt View zurück als AddHelperTaskViewModel mit der EventId und allen HelperTaskTypes &
		        	int eventId &
		        	Planer
		        	\tabularnewline

		        	/HelperTask/Create &
		        	POST &
		        	Speichert den neu erstellten HelperTask in DB, ausser wenn ModelState nicht Valid ist. &
		        	AddHelperTaskViewModel helperTaskViewModel &
		        	Planer
		        	\tabularnewline

		        	/HelperTask/Details/{id} &
		        	GET &
		        	Gibt Helpertask als HelperTaskDetailsViewModel zurück &
		        	int? id &
		        	Admin, Planer, Member
		        	\tabularnewline

		        	/HelperTask/Edit/{id} &
		        	GET &
		        	Gibt HelperTask durch EditHelperTaskViewModel zurück mit einer Liste aller HelperTaskTypes und aller Member. &
		        	int? id &
		        	Planer
		        	\tabularnewline

		        	/HelperTask/Edit &
		        	POST &
		        	Speichert die Änderungen am HelperTask in DB ausser ModelState ist nicht Valid &
		        	EditHelperTaskViewModel model &
		        	Planer
		        	\tabularnewline

		        	/HelperTask/Delete/{id} &
		        	GET &
		        	Löscht Helfereinsatz aus DB &
		        	int? id &
		        	Planer
		        	\tabularnewline

		        	/HelperTask/SignUp/{id} &
		        	GET &
		        	Fügt ein Member einem Helfereinsatz hinzu. Ergebnis wird als JSON zurückgeliefert, da diese Anfrage über AJAX erstellt wird. Hinzufügen wird als Transaktion ausgeführt. Bei Fehlern wird ein Rollback ausgeführt und Error-Message zurückgegeben. &
		        	int? id &
		        	Planer, Member
		        	\tabularnewline

		        	/HelperTask/SignOff{id} &
		        	GET &
		        	Entfernt einen Member von einem Helfereinsatz. Ergebnis wird als JSON zurückgeliefert. Bei Exception wird Rollback von Transaktion ausgeführt &
		        	int? id &
		        	Planer, Member
		        	\tabularnewline

		        \tableend
		        \end{tabularx} 
		    \end{table}

		\subsubsection{AccountController}
			\begin{table}[H]
		        \tablestyle
		        \tablealtcolored
		        \begin{tabularx}{\textwidth}{p{3cm} p{0.8cm} p{6cm} X p{1.2cm}}
		        \tableheadcolor
		            \tablehead Adresse & 
		            \tablehead HTTP &
		            \tablehead Beschreibung &
		            \tablehead Parameter &
		            \tablehead Rechte \\  
		        \tablebody
		        	/Account/Login &
		        	GET &
		        	Gibt Login View zurück &
		        	string returnUrl &
		        	Anonymous
		        	\tabularnewline

		        	/Account/Login &
		        	POST &
		        	Meldet User über Identity an &
		        	LoginViewModel model, string returnUrl &
		        	Anonymous
		        	\tabularnewline

		        	/Account/LogOff &
		        	POST &
		        	Meldet User ab &
		        	- &
		        	-
		        	\tabularnewline

		        	/Account/Manage &
		        	GET &
		        	Gibt View zur Änderung von Benutzersettings zurück &
		        	ManageMessageId? Message &
		        	-
		        	\tabularnewline

		        	/Account/Manage &
		        	POST &
		        	Speichert Änderungen an Benutzer, falls ModelState gültig ist &
		        	ChangeMemberPasswordViewModel model &
		        	-
		        	\tabularnewline

		        	/Account/Index &
		        	GET &
		        	Zeigt View aller Benutzer an &
		        	- &
		        	Admin
		        	\tabularnewline

		        	/Account/Edit/{userid} &
		        	GET &
		        	Zeigt View zur Bearbeitung eines Users an. Gibt Ergebnis als EditMemberViewModel zurück und zusätzlich eine Liste aller Teams und Skills um die Auswahl in der View anzuzeigen &
		        	string id, ManageMessageId? Message = null &
		        	Admin
		        	\tabularnewline

		        	/Account/Edit/ &
		        	POST &
		        	Speichert Änderungen an Benutzer, falls ModelState gültig ist &
		        	EditMemberViewModel model &
		        	Admin
		        	\tabularnewline

		        	/Account/Create &
		        	GET &
		        	Gibt EditMemberViewModel zurück mit Liste aller Teams und Skills m die Auswahl in der View anzuzeigen. Der Member ist hier allerding null &
		        	ManageMessageId? Message = null &
		        	Admin
		        	\tabularnewline

		        	/Account/Create &
		        	POST &
		        	Speichert neuen User ausser ModelState ist nicht gültig &
		        	EditMemberViewModel model &
		        	Admin
		        	\tabularnewline

		        	/Account/MemberRoles/{userId} &
		        	GET &
		        	Zeigt View zur Bearbeitung der Rolle eines Members. Gibt MemberRolesViewModel zurück mit Liste aller verfügbaren Rollen. &
		        	string id &
		        	Admin
		        	\tabularnewline

		        	/Account/MemberRoles &
		        	POST &
		        	Speichert die neuen Memberrollen ausser ModelState ist nicht valid &
		        	MemberRolesViewModel model &
		        	Admin
		        	\tabularnewline

		        \tableend
		        \end{tabularx} 
		    \end{table}

		\subsubsection{ApiController}
			
			\begin{table}[H]
		        \tablestyle
		        \tablealtcolored
		        \begin{tabularx}{\textwidth}{p{3cm} p{0.8cm} p{6cm} X p{1.2cm}}
		        \tableheadcolor
		            \tablehead Adresse & 
		            \tablehead HTTP &
		            \tablehead Beschreibung &
		            \tablehead Parameter &
		            \tablehead Rechte \\  
		        \tablebody
		        	Api/Events/{id} &
		        	GET &
		        	Gibt Event mit ID zurück &
		        	int id &
		        	Admin, Planer, Member
		        	\tabularnewline

		        	Api/Events/{id} &
		        	DELETE &
		        	Löscht Event mit ID &
		        	int id &
		        	Planer
		        	\tabularnewline

		        	Api/HelperTasks &
		        	POST &
		        	Erstellt oder bearbeitet einen Helfereinsatz &
		        	HeleprTaskDetailsViewModel helperTaskViewModel &
		        	Planer
		        	\tabularnewline

		        	Api/HelperTasks/{id} &
		        	DELETE &
		        	Löscht einen Helfereinsatz mit der angegebenen ID &
		        	int? Id &
		        	Planer
		        	\tabularnewline

		        	Api/HelperTaskTypes/ &
		        	GET &
		        	Gibt alle HelperTaskTypes zurück &
		        	- &
		        	Admin, Planer, Member
		        	\tabularnewline

		        	Api/HelperTaskTypes/{id} &
		        	GET &
		        	Gibt bestimmten HelperTaskType zurück &
		        	int id &
		        	Admin, Planer, Member
		        	\tabularnewline

		        	Api/HelperTaskTypes &
		        	POST &
		        	Editiert einen HelperTaskType oder erstellt ihn falls keine Id angegeben &
		        	HelperTaskTypeEditViewModel helperTaskTypeEditViewModel &
		        	Planer
		        	\tabularnewline

		        	Api/HelperTaskTypes/{id} &
		        	DELETE &
		        	Löscht angegeben HelperTaskType &
		        	int id &
		        	Planer
		        	\tabularnewline

		        	Api/Members/{username} &
		        	DELETE &
		        	Löscht angegeben User &
		        	string username &
		        	Admin
		        	\tabularnewline

		        	Api/Skills &
		        	GET &
		        	Gibt alle Skills zurück &
		        	- &
		        	Admin, Planer, Member
		        	\tabularnewline

		        	Api/Skills/{id} &
		        	GET &
		        	Gibt angegeben Skill zurück &
		        	int id &
		        	Admin, Planer, Member
		        	\tabularnewline

		        	Api/Skills &
		        	POST &
		        	Erstellt oder ändert Skills &
		        	SkillEditViewModel skillEditViewModel &
		        	Admin
		        	\tabularnewline

		        	Api/Skills &
		        	DELETE &
		        	Löscht angegeben Skill &
		        	int id &
		        	Admin
		        	\tabularnewline
		        \tableend
		        \end{tabularx} 
		    \end{table}

	\subsection{ASP.NET Identity (Authentifizierung und Authentisierung)}
		Wir verwenden ASP.NET Identity in Version 2.0 für die Authentifizierung und Authentisierung. Folgende benutzerrelevanten Funktionen werden vom Projekt unterstützt:
		\\\begin{itemize}	
			\item Login / Logout
			\item Benutzer anlegen, importieren, bearbeiten und löschen (Admin)
			\item Rollenzuweisung
		\end{itemize}
		Diese Funktionalitäten sind im \textit{AccountController.cs} abgehandelt. Die Logik für die obigen Funktionen sind in der Klasse \textit{UserManager.cs}, welche von \textit{ASP.Net Identity} implementiert ist. Die benutzerrelevanten Funktionen sind daher Indirektionen auf den \textit{UserManager}.

		\subsubsection{Rollen}
			\begin{itemize}
				\item Member
				\item Planer
				\item Admin
			\end{itemize}
			Die Rollen werden für die Authentisierung für Controller-Actions und Views benötigt.
			\begin{lstlisting}[language=CSharp, caption=AccountController.cs, label=lst:accountcontroller, firstnumber=1]
// GET: /Acount/Edit/madminis
[Authorize(Roles = "Admin")]
public ActionResult Edit(string id, ManageMessageId? Message = null) {}
			\end{lstlisting}

			Authentisierung in View:
			\begin{lstlisting}[language=CSharp, caption=\_Layout.cshtml, label=lst:layoutauthentisierung, firstnumber=1]
@if (User.IsInRole("Admin") || User.IsInRole("Admin"))
			\end{lstlisting}

		\subsubsection{Passwortverschlüsselung}
		Das Passwort wird als Hash (\textit{HMACSHA256}) in der Datenbank abgespeichert. Das Projekt verwendet die Standardimplementation von Asp.Net Idenity mit \textit{PasswordHasher}. Der \textit{PasswordHasher} verwendet die \href{http://de.wikipedia.org/wiki/PBKDF2}{\textit{PBKDF2}} um den Hash inkl. einen Saltwert zu strecken. Die Streckung wird 5000 mal durchgeführt. Diese Verkettung erschwert es, per Brute-Force-Methode aus dem Schlüssel auf das ursprüngliche Passwort zu schliessen.
		\\ Das Passwort hat folgende Anforderungen und ist in \textit{MemberRepository.cs} konfiguriert:

		\begin{lstlisting}[language=CSharp, caption=MemberRepository.cs, label=lst:memberrepositorypassword, firstnumber=1]
UserManager.PasswordValidator = new PasswordValidator
{
    RequiredLength = 8,
    RequireNonLetterOrDigit = false,
    RequireDigit = true,
    RequireLowercase = true,
    RequireUppercase = true
};
		\end{lstlisting}