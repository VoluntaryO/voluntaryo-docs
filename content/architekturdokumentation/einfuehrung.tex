%!TEX root = ../../architekturdokumentation.tex
\chapter{Einführung}
	\section{Zweck}
	Dieses Dokument beschreibt die Architektur, die grundlegenden Ideen, Konzepte und Überlegungen für das Projekt VoluntaryO. Es dient den beteiligten Entwicklern die Aspekte des Systems zu definieren und zu verstehen können.
	Nachfolgend werden die relevanten Eigenschaften der Architektur festgehalten. Essentielle Konzepte und Aspekte werden für die Entwickler dokumentiert und beschrieben.
	
	\section{Übersicht}
	Das System VoluntaryO kann in drei Schichten unterteilt werden. Die Datenbank, Dienste wie die Importfunktionen der Webservices und die Präsentationsschicht für die Anzeige auf dem Client.
	Die Datenbank wird mittels Entity Framework und dem Code First Verfahren erstellt. Das Frontend wird nach dem MVC Patter und mit ASP.NET realisiert. Der Service-Layer soll für die verschiedensten Dienste der Applikation zuständig sein.
	Das Deployment findet direkt aus dem Visual Studio auf den Server statt.

	\section{Abgrenzung}
	Das Kapitel "Technologische Übersicht" soll der einfachen Übersicht dienen. Daher wurden für die Darstellung der groben Konzepte keine Standards eingehalten sondern vor allem auf die einfache Verständlichkeit geachtet.
