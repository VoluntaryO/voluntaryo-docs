%!TEX root = ../../architekturdokumentation.tex
\chapter{Architektonische Ziele und Einschränkungen}
	\section{Safety / Security)}
	Auf die Applikation kann nur über HTTP zugegriffen werden. Eine Verschlüsselung ist nicht vorhergesehen, da keine kritischen Daten übermittelt werden. Die Applikation wird mit den gängigen Empfehlungen von Microsoft entwickelt, um die Sicherheit grundsätzlich sicherzustellen.
	Passwörter werden nicht im Klartext gespeichert. Um Änderungen an dem Datenbestand vorzunehmen müssen die Benutzer angemeldet sein.
	Auf der Datenbankebene wird kein Rechtesystem implementiert, da diese nicht direkt ansprechbar ist. Daher wird die ganze Rechtevergabe über den Applikationscode implementiert.
\newcommand{\specialcell}[2][c]{%
  \begin{tabular}[#1]{@{}c@{}}#2\end{tabular}}

	\section{Tools}	
		\begin{table}[H]
		    \tablestyle
		    \tablealtcolored
		    \begin{tabularx}{\textwidth}{l X}
		        \tableheadcolor
		            \tablehead Bezeichnung &
		            \tablehead Erläuterung \tabularnewline
		        \tablebody
		        \textbf{Visual Studio 2013} &
		        	Entwicklungsumgebung, Quellcodeverwaltung
		            \tabularnewline
		        \textbf{JetBrains ReSharper 8.1} &
		            Toolset für Visual Studio 2013, Test Coverage, Coding Standards
		            \tabularnewline
		        \textbf{Astah Community 6.8.0} &
		            Erstellung von Diagrammen
		            \tabularnewline
		        \textbf{Visual Studio Online} &
		            Ablage der Codebase mittels TFS, Builds
		            \tabularnewline
		        \textbf{SQL Management Studio 2012} &
		            Zugriff und Verwaltung der MSSQL Datenbank
		            \tabularnewline
		        \textbf{\LaTeX} &
		            Textsetzung für Dokumentationen
		            \tabularnewline
		        \textbf{Git / GitHub} &
		            Ablage der Dokumentationen 
		            \tabularnewline
		        \textbf{\href{http://www.minutes.io}{minutes.io}} &
		            Tool für Besprechungsnotizen
		            \tabularnewline
		        \tableend
		    \end{tabularx}
		    \caption{Verwendete Tools}
		\end{table}

	\section{Komponenten}	
		\begin{table}[H]
		    \tablestyle
		    \tablealtcolored
		    \begin{tabularx}{\textwidth}{p{3cm} X r}
		        \tableheadcolor
		            \tablehead Bezeichnung &
		            \tablehead Erläuterung &
		            \tablehead Version \tabularnewline
		        \tablebody
			        \textit{Entity Framework} & Abstraktion des Datenbankzugriffs & 6.1 \tabularnewline
			        \textit{Generic Unit of Work \& Repositories Framework)} &  Ermöglicht generische Repositories mit zugehörigen Service-Klassen und eine Unit of Work. Beschreibung in Kaptiel \textit{VoluntaryO.Dal}. & 3.3 \tabularnewline
			        \textit{Effort} & ADO.Net Provider welcher eine In-Memory (temporär!) Datenbank für das EF zur Verfügung stellt. Dies wird für die Unit Tests genutzt. & 1.0.0-beta5 \tabularnewline
			        \textit{log4net} & Zum loggen von Meldungen aus der Appliktation heraus & 1.2.13 \tabularnewline
			        \textit{Unity} & \textit{Dependency Injection} Container für VoluntaryO.Web & 3.5 \tabularnewline
			        \textit{Json.NET} & Parsen der Webschnittstellen in VoluntaryO.Service & 6.0.0 \tabularnewline
			        \textit{Bootstrap} & HTML/CSS Framework für Responsive Design in VoluntaryO.Web & 3.1 \tabularnewline
			        \textit{jQuery} & JavaScript Bibliothek für VoluntaryO.Web & 1.10.2 \tabularnewline
			        \textit{jQuery Choosen} & Benutzerfreundliche Auswahlboxen VoluntaryO.Web & 1.1.0 \tabularnewline
			        \textit{Bootstrap Datetimepicker} & Benutzerfreundliche Datumsauswahl VoluntaryO.Web & 3.0.0 \tabularnewline
		        \tableend
		    \end{tabularx}
		    \caption{Verwendete Komponenten}
		\end{table}
