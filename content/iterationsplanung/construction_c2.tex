%!TEX root = ../../iterationsplanung.tex
\chapter{Construction C2}
	\section{Planung}
    \begin{table}[H]
        \tablestyle
        \tablealtcolored
        \begin{tabularx}{\textwidth}{l X l l r}
        \tableheadcolor
            \tablehead Key &
            \tablehead Summary & 
            \tablehead Priority &
            \tablehead Component/s &
            \tablehead Estimate [h] \tabularnewline  
        \tablebody
			VOLO-104 & Architekturdokumentation fertigen          & Major & Implementation      & 8 \tabularnewline
			VOLO-92  & Integrationstest C2                        & Major & Qualitätsmanagement & 6 \tabularnewline
			VOLO-83  & Resultate aus Review MS5 implementieren    & Major & Projektmanagement   & 3 \tabularnewline
			VOLO-82  & MS5 Architektur/Design                     & Major & Projektmanagement   & 4 \tabularnewline
			VOLO-65  & Reserve Bugfixing C2                       & Major & Qualitätsmanagement & 6 \tabularnewline
			VOLO-68  & Projektsitzung C2                          & Major & Projektmanagement   & 8 \tabularnewline
			VOLO-106 & Erinnerungsfunktion via E-Mail             & Major & Implementation      & 8 \tabularnewline
			VOLO-107 & Reserve für C2                             & Major & Implementation      & 8 \tabularnewline
			VOLO-111 & Code Review C2 für wesentliche Komponenten & Major & Qualitätsmanagement & 8 \tabularnewline
			VOLO-114 & Systemtests (Use Cases)                    & Major & Qualitätsmanagement & 6 \tabularnewline
		    \bottomrule
		    \multicolumn{4}{l}{\textbf{Summe}} & 65 \tabularnewline
        \tableend
        \end{tabularx} 
    \end{table}		
	
	\section{Realisierung}
	
	\section{Erkenntnisse}