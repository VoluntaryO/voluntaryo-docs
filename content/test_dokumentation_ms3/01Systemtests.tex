\chapter{Systemtests}
	\section{Elaboration E2}
	\subsection{Voraussetzungen}
	Die Voraussetzungen für den Systemtest in der Elaboration Phase 2, sind eine bestehende Datenbank und eine Weboberfläche die mit der Datenbank interagiert. Die Tabellen sind inklusive Zwischentabellen noch statisch generiert.
	\subsection{Vorbereitungen}
	Da in dieser Phase noch keine Arbeiten bezüglich Benutzerverwaltung und Login eingeplant sind, werden die Daten statisch in die Datenbank gespeichert. 
	\subsection{Use Case 01: Für Helfereinsatz anmelden}
	Die Funktionen in dieser Phase beziehen sich hauptsächlich darauf, Daten direkt auf dem Model hinzuzufügen, ändern und löschen zu können.
		\begin{table}[H]
    	\tablestyle
    	\tablealtcolored
    	\begin{tabularx}{\textwidth}{l X X}
        	\tablebody
          	\textbf{1} & Der Benutzer kann den Helfereinsatz auswählen und sich anmelden. & Der Helfereinsatz ist ersichtlich und kann ausgewählt werden. Wird also im Dropdown angezeigt.
            \tabularnewline
        	\textbf{2} & Das System überprüft ob die Sperrfrist schon überschritten wurde & Die Überprüfung der Sperrfrist wird nicht funktionieren. Die Implementierung ist erst später eingeplant. 
            \tabularnewline
            \textbf{3} & Der Benutzer kann sich von einem angemeldeten Einsatz wieder abmelden. & Das Abmelden erfolgt durch das erfolgreiche Löschen des Eintrages aus der MemberTaskMapping Tabelle. Der Eintrag kann durch die Implementation der CRUD-Funktionen gelöscht werden.   
            \tabularnewline
            \textbf{4} & Das System überprüft, ob die Anforderungen für den Helfereinsatz erfüllt werden & Es werden zu diesem Zeitpunkt im Projekt noch keine Anforderungen überprüft, deshalb wird es noch keine Fehlermeldung geben.   
            \tabularnewline
           	\tableend
    	\end{tabularx}
   		\caption{UC01: Für Helfereinsatz an-/abmelden}
	\end{table}
	
	\subsection{Testprotokoll UC01}
	\begin{table}[H]
    	\tablestyle
    	\tablealtcolored
    	\begin{tabularx}{\textwidth}{l X l}
        	\tablebody
          	\textbf{1} & Der Benutzer kann sich  & \textcolor{green}{OK}
            \tabularnewline
        	\textbf{2} & Überprüfung findet nicht statt, Verhalten gemäss vorhersage. & \textcolor{green}{OK}
            \tabularnewline
            \textbf{3} & Das Mapping zwischen Benutzer und Helfereinsatz kann gelöscht werden. & \textcolor{green}{OK} 
            \tabularnewline
            \textbf{4} & Anforderungen zum Helfereinsatz werden noch nicht überprüft, Verhalten gemäss Vorhersage. & \textcolor{green}{OK} 
            \tabularnewline
           	\tableend
    	\end{tabularx}
   		\caption{Testprotokoll: UC01}
	\end{table}
	

		\subsection{Use Case 03: Helfereinsatz planen}
		\begin{table}[H]
    	\tablestyle
    	\tablealtcolored
    	\begin{tabularx}{\textwidth}{l X X}
        	\tablebody
          	\textbf{1} & Ein neuer Helfereinsatz kann erstellt werden & Informationen können zum Helfereinsatz können eingetragen werden.
            \tabularnewline
        	\textbf{2} & Der Helfereinsatz kann einem bestehenden Event zugewiesen werden & Anstehende Events werden mittels Dropdown Fenster angezeigt und können ausgewählt werden.
            \tabularnewline
            \textbf{3} &  & Detailansicht des Helfereinsatzes gibt Informationen über Eventname, Helfereinsatztyp, Helfereinsatzkommentar, Start-/Endzeit, maximale Anzahl Personen und die dafür angemeldeten Benutzer.   
            \tabularnewline
            \textbf{4} & Helfereinsatz kann bearbeitet werden & Die Informationen des Helfereinsatzes können bearbeitet werden.
            \tabularnewline
           	\tableend
    	\end{tabularx}
   		\caption{UC03: Helfereinsatz planen}
	\end{table}
		\subsection{Testprotokoll UC03}
	\begin{table}[H]
    	\tablestyle
    	\tablealtcolored
    	\begin{tabularx}{\textwidth}{l X l}
        	\tablebody
          	\textbf{1} & Nach dem Klicken auf "Create new HelperTask" erscheint die Maske um einen Helfereinsatz zu erstellen & \textcolor{green}{OK}
            \tabularnewline
        	\textbf{2} & Das Dropdown Fenster erscheint und ein vorhandener Event kann ausgewählt werden & \textcolor{green}{OK}
            \tabularnewline
            \textbf{3} & In der Detailansicht werden bis auf die dafür eingetragenen Benutzer alle Informationen dargestellt & \textcolor{orange}{NOK} 
            \tabularnewline
            \textbf{4} & Informationen können verändert werden. Die Änderungen werden anschliessend übernommen. Benutzer können noch nicht nachträglich hinzugefügt und entfernt werden. & \textcolor{orange}{NOK} 
            \tabularnewline
           	\tableend
    	\end{tabularx}
   		\caption{Testprotokoll: UC03}
	\end{table}
	
	
	\section{Elaboration E3}
	\subsection{Voraussetzungen}
	Die Voraussetzungen für den Systemtest in der End of Elaboration Phase sind eine bestehende Datenbank und eine Weboberfläche die mit der Datenbank interagiert.
	\subsection{Vorbereitungen}
	Die Hauptfunktionalitäten sind in dieser Phase grösstenteils implementiert. Das Projekt wird nun in unterschiedlichen Schichten unterteilt in denen Daten und Funktionalität klar getrennt werden können. 
	\subsection{Use Case 01: Für Helfereinsatz anmelden}
		\begin{table}[H]
    	\tablestyle
    	\tablealtcolored
    	\begin{tabularx}{\textwidth}{l X X}
        	\tablebody
          	\textbf{1} & Der Benutzer kann den Helfereinsatz auswählen und sich anmelden. & Der Helfereinsatz ist ersichtlich und kann ausgewählt werden. Wird also im Dropdown angezeigt.
            \tabularnewline
        	\textbf{2} & Das System überprüft ob die Sperrfrist schon überschritten wurde & Die Überprüfung der Sperrfrist wird nicht funktionieren. Die Implementierung ist erst später eingeplant. 
            \tabularnewline
            \textbf{3} & Der Benutzer kann sich von einem angemeldeten Einsatz wieder abmelden. & Das Abmelden erfolgt durch das erfolgreiche Löschen des Eintrages aus der MemberTaskMapping Tabelle. Der Eintrag kann durch die Implementation der CRUD-Funktionen gelöscht werden.   
            \tabularnewline
            \textbf{4} & Das System überprüft, ob die Anforderungen für den Helfereinsatz erfüllt werden & Es werden zu diesem Zeitpunkt im Projekt noch keine Anforderungen überprüft, deshalb wird es noch keine Fehlermeldung geben.  
              \tabularnewline
            \textbf{5} & Eigene Angemeldete Helfereinsätze können angezeigt werden & Die View für diese Ansicht besteht schon, jedoch ist diese Ansicht noch nicht komplett implementiert.
              \tabularnewline
            \textbf{6} & Die angemeldeten Helfereinsätze können exportiert werden & Zu diesem Zeitpunkt sind noch keine Arbeitspakete für die Exportfunktion eingeplant. 
            \tabularnewline
           	\tableend
    	\end{tabularx}
   		\caption{UC01: Für Helfereinsatz an-/abmelden}
	\end{table}
	
	\subsection{Testprotokoll UC01}
	\begin{table}[H]
    	\tablestyle
    	\tablealtcolored
    	\begin{tabularx}{\textwidth}{l X l}
        	\tablebody
          	\textbf{1} & Der Helfereinsatz wird angezeigt und kann ausgewählt werden. & \textcolor{green}{OK}
            \tabularnewline
        	\textbf{2} & Überprüfung findet nicht statt, Verhalten gemäss Vorhersage.  Unit Tests dafür sind jedoch schon vorhanden. & \textcolor{green}{OK}
            \tabularnewline
            \textbf{3} & Das Mapping zwischen Benutzer und Helfereinsatz kann gelöscht werden. & \textcolor{green}{OK} 
            \tabularnewline
            \textbf{4} & Anforderungen zum Helfereinsatz werden noch nicht überprüft, Verhalten gemäss Vorhersage. Unit Tests dafür sind jedoch schon vorhanden. & \textcolor{green}{OK} 
                        \tabularnewline
            \textbf{5} & Die View besteht im Moment im MockupEvent  & \textcolor{green}{OK} 
                        \tabularnewline
            \textbf{6} & Anforderungen zum Helfereinsatz werden noch nicht überprüft, Verhalten gemäss Vorhersage & \textcolor{green}{OK} 
            \tabularnewline
           	\tableend
    	\end{tabularx}
   		\caption{Testprotokoll: UC01}
	\end{table}
	

		\subsection{Use Case 03: Helfereinsatz planen}
		\begin{table}[H]
    	\tablestyle
    	\tablealtcolored
    	\begin{tabularx}{\textwidth}{l X X}
        	\tablebody
          	\textbf{1} & Ein neuer Helfereinsatz kann erstellt werden & Informationen können zum Helfereinsatz können eingetragen werden.
            \tabularnewline
        	\textbf{2} & Der Helfereinsatz kann einem bestehenden Event zugewiesen werden & Anstehende Events werden mittels Dropdown Fenster angezeigt und können ausgewählt werden.
            \tabularnewline
            \textbf{3} & Detailierte Informationen zum Helfereinsatz aufrufen & Detailansicht des Helfereinsatzes gibt Informationen über Eventname, Helfereinsatztyp, Helfereinsatzkommentar, Start-/Endzeit, maximale Anzahl Personen und die dafür angemeldeten Benutzer.   
            \tabularnewline
            \textbf{4} & Helfereinsatz kann bearbeitet werden & Die Informationen des Helfereinsatzes können bearbeitet werden.
            \tabularnewline
           	\tableend
    	\end{tabularx}
   		\caption{UC03: Helfereinsatz planen}
	\end{table}
		\subsection{Testprotokoll UC03}
	\begin{table}[H]
    	\tablestyle
    	\tablealtcolored
    	\begin{tabularx}{\textwidth}{l X l}
        	\tablebody
          	\textbf{1} & Nach dem Klicken auf Create new HelperTask erscheint die Maske um einen Helfereinsatz zu erstellen & \textcolor{green}{OK}
            \tabularnewline
        	\textbf{2} & Das Dropdown Fenster erscheint und ein vorhandener Event kann ausgewählt werden & \textcolor{green}{OK}
            \tabularnewline
            \textbf{3} & In der Detailansicht werden bis auf die dafür eingetragenen Benutzer alle Informationen dargestellt & \textcolor{green}{OK} 
            \tabularnewline
            \textbf{4} & Informationen können verändert werden. Die Änderungen werden anschliessend übernommen. Benutzer können noch nicht nachträglich hinzugefügt und entfernt werden. & \textcolor{green}{OK} 
            \tabularnewline
           	\tableend
    	\end{tabularx}
   		\caption{Testprotokoll: UC03}
	\end{table}
	\section{Construction C1}
	\subsection{Voraussetzungen}
	Die Voraussetzungen für den Systemtest in der Construction C1 Phase sind die selben Voraussetzungen wie in den vorherigen Phasen.
	\subsection{Vorbereitungen}
	Die Hauptfunktionalitäten sind in dieser Phase implementiert. Der Schwerpunkt dieser Phase liegt in der korrekten Einteilung unseres Projektes gemäss der geplanten SW-Architektur.
	\subsection{Use Case 01: Für Helfereinsatz anmelden}
		\begin{table}[H]
    	\tablestyle
    	\tablealtcolored
    	\begin{tabularx}{\textwidth}{l X X}
        	\tablebody
          	\textbf{1} & Der Benutzer kann den Helfereinsatz auswählen und sich anmelden. & Der Helfereinsatz ist ersichtlich und kann ausgewählt werden. Wird also im Dropdown angezeigt.
            \tabularnewline
        	\textbf{2} & Das System überprüft ob die Sperrfrist schon überschritten wurde & \textcolor{green}{implementation noch offen} 
            \tabularnewline
            \textbf{3} & Der Benutzer kann sich von einem angemeldeten Einsatz wieder abmelden. & Der Benutzer kann nun durch den "Abmelden"-Button sich vom Helfereinsatz abmelden.  
            \tabularnewline
            \textbf{4} & Das System überprüft, ob die Anforderungen für den Helfereinsatz erfüllt werden & Überprüfung der Skills wird nun durchgeführt und die Einsätze werden nur zugewiesen, wenn alle Anforderungen erfüllt werden können.  
              \tabularnewline
            \textbf{5} & Angemeldete Helfereinsätze können angezeigt werden & Die eingenen Helfereinsätze können angezeigt und verwaltet werden.
              \tabularnewline
            \textbf{6} & Die angemeldeten Helfereinsätze können exportiert werden & Zu diesem Zeitpunkt sind noch keine Arbeitspakete für die Exportfunktion eingeplant. 
            \tabularnewline
           	\tableend
    	\end{tabularx}
   		\caption{UC01: Für Helfereinsatz an-/abmelden}
	\end{table}
	
	\subsection{Testprotokoll UC01}
	\begin{table}[H]
    	\tablestyle
    	\tablealtcolored
    	\begin{tabularx}{\textwidth}{l X l}
        	\tablebody
          	\textbf{1} & Der Helfereinsatz wird angezeigt und kann ausgewählt werden. & \textcolor{green}{OK}
            \tabularnewline
        	\textbf{2} & Überprüfung findet nicht statt, Verhalten gemäss Vorhersage.  Unit Tests dafür sind jedoch schon vorhanden. & \textcolor{green}{OK}
            \tabularnewline
            \textbf{3} & Der Benutzer kann sich normal über die Weboberfläche vom Helfereinsatz abmelden & \textcolor{green}{OK} 
            \tabularnewline
            \textbf{4} & Die Anmeldung wird nur ausgeführt, wenn alle Anforderungen vom Mitglied erfüllt werden können & \textcolor{green}{OK} 
                        \tabularnewline
            \textbf{5} & Die eigenen Helfereinsätze können nun angezeigt und verwaltet werden. & \textcolor{green}{OK} 
                        \tabularnewline
            \textbf{6} & Anforderungen zum Helfereinsatz werden nun überprüft. & \textcolor{green}{OK} 
            \tabularnewline
           	\tableend
    	\end{tabularx}
   		\caption{Testprotokoll: UC01}
	\end{table}
	

		\subsection{Use Case 03: Helfereinsatz planen}
		\begin{table}[H]
    	\tablestyle
    	\tablealtcolored
    	\begin{tabularx}{\textwidth}{l X X}
        	\tablebody
          	\textbf{1} & Ein neuer Helfereinsatz kann erstellt werden & Informationen können zum Helfereinsatz können eingetragen werden.
            \tabularnewline
        	\textbf{2} & Der Helfereinsatz kann einem bestehenden Event zugewiesen werden & Anstehende Events werden mittels Dropdown Fenster angezeigt und können ausgewählt werden.
            \tabularnewline
            \textbf{3} & Detailierte Informationen zum Helfereinsatz aufrufen & Detailansicht des Helfereinsatzes gibt Informationen über Eventname, Helfereinsatztyp, Helfereinsatzkommentar, Start-/Endzeit, maximale Anzahl Personen, Mindestanforderungen und die dafür angemeldeten Benutzer.   
            \tabularnewline
            \textbf{4} & Helfereinsatz kann bearbeitet werden & Die detailierten Informationen des Helfereinsatzes können bearbeitet werden.
            \tabularnewline
           	\tableend
    	\end{tabularx}
   		\caption{UC03: Helfereinsatz planen}
	\end{table}
		\subsection{Testprotokoll UC03}
	\begin{table}[H]
    	\tablestyle
    	\tablealtcolored
    	\begin{tabularx}{\textwidth}{l X l}
        	\tablebody
          	\textbf{1} & Neue Helfereinsätze können nach der Auswahl eines Events erstellt werden. & \textcolor{green}{OK}
            \tabularnewline
        	\textbf{2} & Der Test ist erfolgreich, durch die Umstrukturierung in Test 1. & \textcolor{green}{OK}
            \tabularnewline
            \textbf{3} & Alle Informationen konnten abgerufen werden. & \textcolor{green}{OK} 
            \tabularnewline
            \textbf{4} & Informationen können verändert werden. Die Änderungen werden anschliessend übernommen. \textcolor{orange}{Benutzer können jedoch noch nicht nachträglich hinzugefügt oder entfernt werden.} & \textcolor{orange}{NOK} 
            \tabularnewline
           	\tableend
    	\end{tabularx}
   		\caption{Testprotokoll: UC03}
	\end{table}