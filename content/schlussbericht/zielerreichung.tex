%!TEX root = ../../schlussbericht.tex
\chapter{Zielerreichung}
	
	
	\section{Erfüllung der Requirements}	
	Wir konnten unser Projekt fertigstellen und somit unser Ziel erreichen. Die von uns gestellten Anforderungen in der Anforderungsspezifikation konnten erfüllt werden.
	
	\begin{table}[H]
        \tablestyle
        \tablealtcolored
        \begin{tabularx}{\textwidth}{X X}
        \tableheadcolor
            \tablehead Funktion & 
            \tablehead Ort \\  
        \tablebody
            \textbf{F01} Planung der Helfereinsätze & 
            /Event/Edit/\{id\}  \tabularnewline
            
            \textbf{F02} Unterteilung des Einsatz nach Zeit und Funktion & 
            /Event/Edit/\{id\}  \tabularnewline
            
             \textbf{F03} Importieren der Spieldaten von Webservice & 
            /EventImport  \tabularnewline
            
              \textbf{F04} Manuelles Erfassen von Spieldaten & 
            /Event/Create  \tabularnewline
            
              \textbf{F05} Importieren der Endbenutzer aus webling.ch & 
            /MemberImport  \tabularnewline
            
              \textbf{F06} Manuelles Erfassen von Mitglied & 
            /Account/Create  \tabularnewline
            
              \textbf{F07} Einteilung der Einsätze nach Mannschaft & 
            /Event/Edit/\{id\}  \tabularnewline
            
            \textbf{F08} Einschreiben eines Mitglieds für Einsatz & 
            /Event/Details/\{id\} \newline /HelperTask/Details/\{id\}  \tabularnewline
            
            \textbf{F09} Austragen eines Mitglieds von Einsatz innerhalb Frist & 
            /Event/Details/\{id\} \newline /HelperTask/Details/\{id\}  \tabularnewline
            
            
            \textbf{F10} Übersicht der Einsätze für Mitglied & 
            \textit{Startseite}  \tabularnewline
            
            \textbf{F11} Erinnerung der Helfer vor Einsatz via E-Mail & 
            -  \tabularnewline
            
            
            
            
            
            

				
			         
        \tableend
        
        \end{tabularx} 
    \end{table}
	
	\section{Nicht realisierte Features}
	
	
	\section{Kosten und Zeit}
	

	\section{Aufwand pro Aktivität}
	
	\section{Aktivität pro Teammitglied}
	
	\section{Code-Qualität}
	