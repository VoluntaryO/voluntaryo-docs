%!TEX root = ../../schlussbericht.tex
\chapter{Allgemeiner Erfahrungsbericht}
	\section{Team und Organisation}
	
		\subsection{Teamspirit}
		Da alle Teammitglieder schon vor Projektstart miteinander befreundet waren, konnten wir  von Beginn an auf die verschiedenen Stärken der jeweiligen Personen eingehen. Die Motivation und die Lust aufs Lernen waren hoch. Wer an seine Grenzen stiess konnte sich stets auf die Hilfe der anderen Teammitglieder verlassen. Alle hatten den Ehrgeiz sich den neuen Herausforderungen zu stellen und haben sich auch ausserhalb ihrer Komfortzonen bewegt.\\ \\
		Philipp Meier hat als \textit{Primus inter Pares} die Koordination der Aufgaben hervorragend unterstützt. 
		
		\subsection{Kommunikation}
		Die Kommunikation im Team war für das Gelingen unseres Projekts sehr entscheidend. Besonders in den Inception- und Elaboration-Phasen war es von enormer Wichtigkeit, dass sich das ganze Team oft zusammensetzte, um einen direkten Austausch zu ermöglichen. Dabei entstanden viele sinnvolle und interessante Diskussionen, von denen die Construction-Phasen massgeblich beeinflusst worden sind.\\ \\
		Bei der Implementierung der Arbeitspakete wurde vermehrt auch selbstständig gearbeitet. Dabei tauschten wir uns mit Hilfe von elektronischen Nachrichten über das aktuelle Vorankommen aus. Wir reservierten uns jeweils mindestens einen Tag in der Woche, an dem sich alle Mitglieder traffen. Dabei stellten wir uns gegenseitig die Umsetzung der Arbeitspakete vor, führten Code Reviews durch, verteilten Aufgaben und halfen einander bei Unklarheiten.
		
		\subsection{Protokolle}
		Zu jeder Iteration wurde entweder ein wöchentliches Meeting-Protokoll oder ein Protokoll zum Review mit dem Betreuer verfasst. Darin wurden die wichtigsten Informationen, Entscheidungen und Aufgaben festgehalten. Die Protokolle waren insbesondere für das Erstellen neuer Tasks im JIRA sinnvoll und dienten auch als Anreiz, sich Gedanken über vorherige Entscheidungen zu machen. 
		
		\subsection{Dokumente}
		Alle Textdokumente wurden mit Hilfe von \LaTeX\ erstellt. Der Initialaufwand zum Anlegen neuer Dokumente ist etwas höher als bei herkömmlichen Textverarbeitungsprogrammen. Für technisch versierte Anwender tun sich dafür viele Vorteile auf. Für die textbasierten .tex-Dokumente eignete sich \textit{Git} hervorragend zur Versionskontrolle. Dadurch konnten wir alle gleichzeitig daran arbeiten, ohne uns über verheerende Konflikte oder Datenverluste zu sorgen.
		
	\section{Qualität}
	\subsection{Projektgrösse}
	\begin{table}[H]
        \tablestyle
        \tablealtcolored
        \begin{tabularx}{\textwidth}{X r r}
        \tableheadcolor
            \tablehead Project & 
            \tablehead Maintainability Index & 
            \tablehead Lines of Code \\
        \tablebody
            VoluintaryO.Dal & 86 & 742 \\
            VoluntaryO.Dal.Test & 69 & 220 \\
            VoluntaryO.Service & 89 & 541 \\
            VoluntaryO.Service.Test & 79 & 400 \\
            VoluntaryO.Mail & 93 & 3 \\
            VoluntaryO.Web & 82 & 1105 \\
            {\bf Total} & \O  83 & 3011  
            \tabularnewline
        \tableend
        \end{tabularx} 
    \end{table}
    \subsection{Code Reviews}
Die wöchentlichen Code Reviews stellten sich für uns als sehr wichtig heraus. Es war für uns nämlich sehr schwer das Know-How aller Teammitglieder auf einem Level zu halten. Besonders wichtig war es die Gedanken und Motivationen hinter den einzelnen Verfahren zu erklären, damit falls von anderen darauf zugegriffen werden muss, diese auch das Verständnis dafür haben.

Beim Code-Review konnte auch frühzeitig eingegriffen werden, falls einzelne Arbeitspakete nicht so implementiert wurden, wie in den Anforderungen vorgesehen.
    
    \subsection{Frameworks und Patterns}
Innerhalb unseres Projekts haben wir einige Patterns und Frameworks eingesetzt, die uns halfen, die Code Qualität zu erhöhen. Besonders herauszuheben sind hier das Unit of Work, Repository und Dependency Injection. Hier benutzen wir verschiedene Frameworks um dies Patterns generisch zu implementieren. Am wichtigsten für die hohe Abstraktion die wir erhalten haben waren die Layers die wir erstellt hatten.
    
    \subsection{Visual Studio, ReSharper und dotCover}
    Während der Entwicklung half uns der ReSharper sehr unseren Codestyle auf einen Nenner zu kriegen. Die nützlichen Tipps und Hilfen, die der ReSharper bietet, brachten uns automatisch zu einem einheitlichen Coding Styles.
    
    Die Analysen und Metriken, die uns Visual Studio und IntelliJ’s dotCover boten, halfen uns sicherzustellen, dass eine gute Testcoverage erreicht wurde und dass die Komplexität genug tief blieb.
    
    \subsection{Test Coverage}
Das Projekt umfasst 55 Unit Tests. Wir haben uns entschieden von der Web-Schicht keine Unit-Tests zu erstellen. Die Controller zur Darstellung der Seite werden nicht durch Unit-Tests getestet. Dafür haben wir System-Tests. Der Grund dafür ist, dass der Aufwand für Unit-Tests hier extrem hoch wäre und der Nutzen ziemlich klein.	\\
	\begin{table}Hh]
        \tablestyle
        \tablealtcolored
        \begin{tabularx}{\textwidth}{X r r r}
        \tableheadcolor
            \tablehead Layer & 
            \tablehead Test-Coverage & 
            \tablehead Lines of Code &
            \tablehead Lines of Code (Tests)\\
        \tablebody
            VoluintaryO.Dal & 95.23 \% & 742 & 220\\
            VoluntaryO.Service & 79.14 \% & 541 & 400\\
            VoluntaryO.Web & 0.00 \% & 1105 & 0\\
            {\bf Total} &  58.12 \%
            \tabularnewline
        \tableend
        \end{tabularx} 
    \end{table}