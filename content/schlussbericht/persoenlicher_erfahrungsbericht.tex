%!TEX root = ../../schlussbericht.tex
\chapter{Persönliche Erfahrungsberichte}
	\section{Nhat-Nam Le}
	Das SE2-Projekt war für mich das erste richtige Softwareprojekt, darum war es sehr interessant, lehrreich und voller neuer Erfahrungen. \\ \\
Am lehrreichsten waren für mich die Projektmanagement Aspekte. Ich bin sehr froh die Projektmanagement bezogenen Erfahrungen schon vor der Semesterarbeit oder der Bachelorarbeit gemacht haben zu können. Diese werden auch in Zukunft ganz bestimmt von gutem Nutzen sein.\\ \\
Technisch befand sich unser ganzes Team auf Neuland. Wir wollten diese Chance Nutzen und uns in diese Technologien einarbeiten. Leider war mir die verfügbare Zeit doch zu knapp, um jeweils ein vertieftes Wissen erlangen zu können. Als sehr positive Erfahrung waren für mich auch die Codereviews. Durch den Austausch der Erfahrungen zwischen den Teammitgliedern konnte ich viel profitieren.\\ \\
Da ich zuvor in keinem Softwareprojekt mitgearbeitet habe, war speziell für mich das Programmieren im Team eine wertvolle Erfahrung. Da der Verwendungszweck des Projekts sehr realistisch erscheint, war der grosse Aufwand in das Projekt, meiner Meinung nach, sehr gut investiert.

	
	\section{Robin Bader}
	Ich habe mir einige Ziele für dieses Projekt gesetzt. Besonders wichtig war für mich endlich mal ein eigenes Projekt anzuwenden und herauszufinden, wie gut ich das theoretische erlernte Wissen über das Programmieren und über die SE-Techniken in der Praxis einsetzen kann. Zusätzlich wollte ich mich endlich mal mit .NET und Web-Technologien auseinander setzen, da mich beide dieser Technologien interessieren, ich aber nicht wirklich wusste, was dahinter steckte.\\ \\
Nun, viele Stunden und Schweisstropfen später, kann ich ein Fazit ziehen. Ich bin stolz auf das von uns erreichte Endergebnis. Ich konnte mich während des Projekts technologisch sehr weiterentwickeln. Ich habe grossen Einsatz gegeben um die Technologien so umzusetzen, wie es gedacht ist. Erstaunt hat mich, wie viel Einsatz man zeigen muss, um technologisch immer auf dem gleichen Level zu bleiben wie die anderen. Sobald man dies einmal verpasst muss man viel Zeit investieren um wieder aufzuholen.\\ \\
Von der Projektsicht aus hätte ich nicht gedacht, dass der Einsatz so hoch ist. Anfangs kamen wir recht schnell zu einem soliden Ergebnis. Ich hatte das Gefühl der grösste Teil des Projektes ist bereits erledigt. Erstaunt war ich dann aber, wie gross der Aufwand für die Details dann aber doch noch ist. Für mich hat sich hier die 80/20 Regel gezeigt. Ich verstand erst nicht, wieso Herr Bläser für die End-Of-Elaboration-Phase einen fast fertigen Prototyp wollte. Schlussendlich war ich dann aber froh, dass der Druck schon so früh so hoch war.\\ \\
Die hohe Motivation im Team hat stark dazu beigetragen, das Ziel zu erreichen.


	\section{Philipp Meier}
	Nach ca. 160 Stunden Aufwand für das SE2-Projekt, kann ich und das Team, auf eine sehr erfolgreich und eine stabile Softwarelösung zurückblicken. Rückblickend ergaben sich für mich folgende Komplexitäten:
	\\\begin{itemize}
		\item Keine Erfahrung in Projektmanagement
		\item Koordination und Aufteilung auf 4 Personen
		\item Koordination und Aufteilung auf nur 10 Stunden/Woche
		\item Know-How auf gleiches Level bringen
		\item Zeitschätzung zu Begin des Projekts
	\end{itemize}
	Ich schlüpfte für das Projekt in die Rolle des \enquote{primären} Projekt Managers und wurde darin von den anderen Teammitglieder bestens unterstützt. Es war schwierig passende Arbeitspakete für das Team zu finden und diese noch auf das Arbeitspensum anzupassen. Wir hatten bereits von Anfang an die Projektzeit/Mitglied auf 10 Stunden/Woche geplant, jedoch gestaltete sich gerade zu Begin die Aufteilung schwierig und brachte einige Unsicherheiten hervor. Durch ein Gespräch mit Herr Bläser konnten diese für den Projektverlauf jedoch beseitigt werdne, und die Planung verlief zunnehmend leichter. Ansonsten habe ich die Koordiantion im Team als sehr konstruktiv und zielführend empfunden.
	\\Allgemein konnte vieles aus den Kursen von SE1 und SE2 anwendet werden und man spürte im Projektverlauf auch die Auswirkungen davon. Die \textit{Unit Tests} haben das Refactoring und Code Reviews spürbar vereinfacht und gleichzeit auch für solche Qualitätsmassnahmen motiviert. Die Metriken für Code, welche in SE2 vorgestellt wurden, haben wir auch probeweise verwendet. Die Resultate daraus waren spannend, jedoch benötigten zuerst einigen Intepretationsaufwand.
	\\ Im Team gab es keine Web-Technologie, welche von allen Mitglieder bereichts gekannt wurde. Wir entschieden uns daher für das modern \textit{ASP.NET MVC Framework} von Mircosoft und erhofften uns eine ausführliche Dokumentation, um schnell mit der Technologie vertraut zu werden. 
	Microsoft bietet auf \href{http://www.asp.net/}{www.asp.net} einfache Guides, welche uns im Einstieg und auch bei archetektonischen Herausvorderungen (\textit{Layering}, \textit{Unit of Work} etc.) für \textit{.NET} unterstützten. 
	\\ Am Schluss möchte ich nochmals hervorheben, dass das Projekt sehr viel Spass gemacht hat. Wir konnten defintiv viel lernen, anwenden und Erfahrungen in allen Bereich für die Zukunft (Semesterarbeit und Bachelorarbeit) sammeln.


	\section{Dominik Freier}
	Die praktische Umsetzung eines gesamten Software-Projekts inklusive Planung, Programmierung, Protokollierung, Dokumentation und Abgabe ist etwas völlig anderes als die vorausgehenden theortischen Informationen und unzusammenhängenden Übungen. Deshalb war die Teilnahme am SE2-Projekt eine sehr wertvolle Erfahrung für mich. Man wurde auch mit weniger liebsamen Aufgaben konfrontiert, die nichts desto trotz ihre Berechtigung in einem solchen Projekt haben. Deren Sinn und Zweck konnte man auf diese Weise am eigenen Leib erfahren.\\ \\
	Unsere Wahl des \textit{ASP.NET MVC Frameworks} als technische Grundlage bereue ich keinesfalls. Trotz Einarbeitungszeit und fehlender Erfahrung gelang uns ein zügiger und solider Aufbau der Archtektur. Obwohl uns das Framework und die gute Unterstützung durch Visual Studio  viel Arbeit abnahm, haben wir gelernt, dass man davon keine Wunder erwarten darf und sich trotzdem intensiv mit der Technologie und den Eigenheiten unserer Lösung auseinandersetzen muss.\\ \\
	Wir haben festgestellt, dass die Koordination eines Teams bereits ab einer Anzahl von vier Mitgliedern eine Komplexität annimmt,  die einen erheblichen Mehraufwand verursacht. Dank wöchentlichen Meetings und gemeinschaftlicher Arbeit mit persönlicher Präsenz gelang es uns, den kontinuierlichen Know-How Transfer zu gewährleisten. 