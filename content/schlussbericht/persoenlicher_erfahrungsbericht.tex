%!TEX root = ../../schlussbericht.tex
\chapter{Persönliche Erfahrungsberichte}
	\section{Nhat-Nam Le}
	Das SE2-Projekt war für mich das erste richtige Softwareprojekt, darum war es sehr interessant, lehrreich und voller neuer Erfahrungen. \\ \\
Am lehrreichsten waren für mich die Projektmanagement Aspekte. Ich bin sehr froh die Projektmanagement bezogenen Erfahrungen schon vor der Semesterarbeit oder der Bachelorarbeit gemacht haben zu können. Diese werden auch in Zukunft ganz bestimmt von gutem Nutzen sein.\\ \\
Technisch befand sich unser ganzes Team auf Neuland. Wir wollten diese Chance Nutzen und uns in diese Technologien einarbeiten. Leider war mir die verfügbare Zeit doch zu knapp, um jeweils ein vertieftes Wissen erlangen zu können. Als sehr positive Erfahrung waren für mich auch die Codereviews. Durch den Austausch der Erfahrungen zwischen den Teammitgliedern konnte ich viel profitieren.\\ \\
Da ich zuvor in keinem Softwareprojekt mitgearbeitet habe, war speziell für mich das Programmieren im Team eine wertvolle Erfahrung. Da der Verwendungszweck des Projekts sehr realistisch erscheint, war der grosse Aufwand in das Projekt, meiner Meinung nach, sehr gut investiert.

	
	\section{Robin Bader}
	Ich habe mir einige Ziele für dieses Projekt gesetzt. Besonders wichtig war für mich endlich mal ein eigenes Projekt anzuwenden und herauszufinden, wie gut ich das theoretische erlernte Wissen über das Programmieren und über die SE-Techniken in der Praxis einsetzen kann. Zusätzlich wollte ich mich endlich mal mit .NET und Web-Technologien auseinander setzen, da mich beide dieser Technologien interessieren, ich aber nicht wirklich wusste, was dahinter steckte.\\ \\
Nun, viele Stunden und Schweisstropfen später, kann ich ein Fazit ziehen. Ich bin stolz auf das von uns erreichte Endergebnis. Ich konnte mich während des Projekts technologisch sehr weiterentwickeln. Ich habe grossen Einsatz gegeben um die Technologien so umzusetzen, wie es gedacht ist. Erstaunt hat mich, wie viel Einsatz man zeigen muss, um technologisch immer auf dem gleichen Level zu bleiben wie die anderen. Sobald man dies einmal verpasst muss man viel Zeit investieren um wieder aufzuholen.\\ \\
Von der Projektsicht aus hätte ich nicht gedacht, dass der Einsatz so hoch ist. Anfangs kamen wir recht schnell zu einem soliden Ergebnis. Ich hatte das Gefühl der grösste Teil des Projektes ist bereits erledigt. Erstaunt war ich dann aber, wie gross der Aufwand für die Details dann aber doch noch ist. Für mich hat sich hier die 80/20 Regel gezeigt. Ich verstand erst nicht, wieso Herr Bläser für die End-Of-Elaboration-Phase einen fast fertigen Prototyp wollte. Schlussendlich war ich dann aber froh, dass der Druck schon so früh so hoch war.\\ \\
Die hohe Motivation im Team hat stark dazu beigetragen, das Ziel zu erreichen.


	\section{Philipp Meier}
	
	
	\section{Dominik Freier}